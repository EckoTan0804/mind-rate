\documentclass[a4paper]{scrreprt}

\usepackage[german]{babel}
\usepackage[utf8]{inputenc}
\usepackage[T1]{fontenc}
\usepackage{ae}
\usepackage[bookmarks,bookmarksnumbered]{hyperref}
\usepackage{graphicx}
\usepackage[toc]{glossaries}
\usepackage{float}
\usepackage{textcomp}
\usepackage{listings}
\lstset{upquote=true}

\graphicspath{ {Images/} }
\setcounter{secnumdepth}{5}
\makeglossaries
% \linespread{1.5}

\begin{document}
    \def\code#1{\texttt{#1}}

    \begin{flushright}
        \includegraphics[scale = 0.7]{kit-logo.jpg}\\[0.5cm]
        % \includegraphics[scale = 1]{teco.jpg}
    \end{flushright}
    % \includegraphics[scale = 0.5]{kit-logo.jpg} \hspace{4cm} \includegraphics[scale = 1]{teco.jpg}
    \vspace*{2cm}

    \begin{center} \large

        Praxis der Softwareentwicklung
        \vspace * {1.5cm}

        \textbf{\huge Mind Rate}

        \vspace*{1cm}


        {\Large Ein interaktives System mit Android-Client f\"ur Studien nach Experience-Sampling-Method (ESM)}

        \vspace*{1cm}

        \textbf{\Large Entwurf}
        \vspace*{2cm}

        Shanshan Du, Yi Ge, Renhan Lou, Ruoheng Ma, Haobin Tan
        \vspace*{1cm}

        02. Dezember 2016
        \vspace*{2.5cm}

        Betreuung: Anja Exler, Dr. Andrea Schankin\\[0.5cm]
        Forschungsgruppe TECO: Technology for Pervasive Computing\\[0.5cm]

        Karlsruher Institut für Technologie
    \end{center}
    \thispagestyle{empty}

    \tableofcontents

    \chapter{Allgemeine Struktur}

        Der Entwurf besteht aus zwei Seiten: der Android-App-Seite und der Server-Seite. Sie sind jeweils für die Android-App und das Web-Interface verantwortlich. Zusätzlich soll eine Datenbank auf dem Server laufen, um alle Studien-, Konto- und Probandendaten zu speichern. Diese gehört auch zu der Server-Seite. Die zwei Seiten sollen miteinander durch das HTTP-Protokoll kommunizieren.


        \section{Überblick über die Intekaktion von Android-App und Server}


            \vspace*{1cm}
            \begin{figure}[H]
                % \centering
                \includegraphics[scale = 0.25]{Overview.jpg}
                \caption{Überblick über die Intekaktion von Android-App und Server}
            \end{figure}



    \chapter{Server-Seite}
        Die Server-Seite nutzt eine von Django-Framework veränderte Variante\footnote{\href{https://docs.djangoproject.com/en/1.10/faq/general/\#django-appears-to-be-a-mvc-framework-but-you-call-the-controller-the-view-and-the-view-the-template-how-come-you-don-t-use-the-standard-names}{https://docs.djangoproject.com/en/1.10/faq/general/\#django-appears-to-be-a-mvc-framework-but-you-call-the-controller-the-view-and-the-view-the-template-how-come-you-don-t-use-the-standard-names}} der Modell-Präsentation-Steuerung-Architektur (MVC) als den Architekturstil. Die benötigte Klassen werden als Modelle entworfen. Der Entwurf von Server-Seite wird auf Python angefertigt.

        \section{Modelle}
            \begin{itemize}
                \item \code{class StudyDirector}
                    \begin{itemize}
                        \item \code{studies = []: Study}
                        \item \code{surname}
                        \item \code{firstName}
                        \item \code{mailAddress}
                    \end{itemize}

                    \item \code{class Study}
                        \begin{itemize}
                            \item \code{probands = []: Proband}
                            \item \code{id}
                            \item \code{name}
                            \item \code{beginningDate}
                            \item \code{endDate}
                            \item \code{director: StudyDirector}
                            \item \code{questionnaires = []: Questionnaire}
                        \end{itemize}

                    \item \code{class Proband}
                        \begin{itemize}
                            \item \code{study: Study}
                            \item \code{id}
                            \item \code{birthday}
                            \item \code{occupation}
                            \item \code{gender}
                        \end{itemize}

                    \item \code{class Questionnairer}
                        \begin{itemize}
                            \item \code{questions = []: AbstractQuestion}
                            \item \code{id}
                            \item \code{name}
                            \item \code{study: Study}
                            \item \code{triggerEvent: TriggerEvent}
                            \item \code{answeringTime}
                            \item \code{answers = []: QuestionnaireAnswer}
                        \end{itemize}

                    \item \code{class QuestionnaireAnswer}
                        \begin{itemize}
                            \item \code{submitter: Proband}
                            \item \code{submitTime}
                            \item \code{questionAnswers = []: AbstractQuestionAnswer}
                        \end{itemize}

                    \item \code{class AbstractQuestion(ABC)}
                        \begin{itemize}
                            \item \code{id}
                            \item \code{questionnaire: Questionnaire}
                        \end{itemize}

                    \item \code{class TextQuestion(AbstractQuestion)}
                        \begin{itemize}
                            \item \code{content}
                        \end{itemize}

                    \item \code{class ChoiceQuestion(AbstractQuestion)}
                        \par // Both single choice question and multiple choice question are belonging to this class.
                        \begin{itemize}
                            \item \code{options = \{\}}
                            \item \code{followUpQuestion: AbstractQuestion}
                            \item \code{followUpQuestionTriggerOptions = []}
                        \end{itemize}

                    \item \code{class SteplessScaleQuestion(AbstractQuestion)}
                        \par // Scale questions with steps are effectively just single choice questions; therefore they are not a class.
                        \begin{itemize}
                            \item \code{scaleMin: int}
                            \item \code{scaleMax: int}
                            \item \code{followUpQuestion: AbstractQuestion}
                            \item \code{followUpQuestionTriggerMin: int}
                            \item \code{followUpQuestionTriggerMax: int}
                        \end{itemize}

                    \item \code{class AbstractQuestionAnswer(ABC)}
                        \begin{itemize}
                            \item \code{content}
                        \end{itemize}

                   \item \code{class TextQuestionAnswer(AbstractQuestionAnswer)}
                       \begin{itemize}
                           \item \code{content: string}
                       \end{itemize}

                   \item \code{class MultiChoiceQuestionAnswer(AbstractQuestionAnswer)}
                       \begin{itemize}
                           \item \code{content = []: string}
                       \end{itemize}

                    \item \code{class SingleChoiceQuestionAnswer(AbstractQuestionAnswer)}
                        \begin{itemize}
                            \item \code{content: string}
                        \end{itemize}

                    \item \code{class SteplessScaleQuestionAnswer(AbstractQuestionAnswer)}
                        \begin{itemize}
                            \item \code{content: int}
                        \end{itemize}

            \end{itemize}

        \newpage
        \section{Entity-Relationship-Modell der Datenbank}
            \begin{figure}[ht]
                \centering
                \includegraphics[scale = 0.13]{PSE_Datenbank_ERM.jpeg}
                \caption{Entity-Relationship-Modell der Datenbank}
            \end{figure}


    \newpage
    \setlength{\baselineskip}{1.5\baselineskip}{
    \chapter{Android-App}

        Das Entwurfsmuster Model-View-Controller (MVC, englisch für Modell-Präsentation-Steuerung) wird hier angewandet, um einen fleixbler Programmentwurf zu ermöglichen.


        \vspace*{0.5cm}
        \section{Überblick über Pakete}

            Das folgende Diagramm dient sich als einen Überblick über unsere Android App. Modell, Präsentation und Steuerung werden mit drei verschiedenen Farbe dargestellt: blau für Modell, grün für Präsentation und gelb für Steuerung.

            \noindent Um die Beziehungen zwischen Klassen und das System ganzheitlich aufzufassen, enthält folgendes Diagramm keine Attribute und Methoden. Jede Pakete sowie Klasse mit Attribute und (offene) Methoden werden dann ausführlich beschrieben.



            \newpage
            \vspace*{1cm}
            \begin{figure}[H]
                \centering
                \includegraphics[scale = 0.25]{packageDiagram.jpg}
                \caption{Überblick über alle Packte und Klasse (ohne Attribute und Methoden)}
            \end{figure}




        %TODO: Ausführliche Beschreibung für jedes package und klasse davon

        
        \section{Model}

            Der Teil Model besteht aus zwei Pakete:

            \begin{itemize}
                \item package \code{gson}
                    \begin{itemize}
                        \item \code{abstract class QuestionStrategy}
                        \item \code{class SingleChoice}
                        \item \code{class MutipleChoice}
                        \item \code{class StepScale}
                        \item \code{class DragScale}
                        \item \code{class TextAnswer}
                        \item \code{class Proband}
                        \item \code{class Questionnaire}
                        \item \code{class Question}
                    \end{itemize}
                \item package \code{sensor}
                    \begin{itemize}
                        \item \code{SensorManager}
                        \item \code{AccelerometerSensor}
                        \item \code{MagneticFeldSensor}
                        \item \code{GravitySensor}
                        \item \code{TemperatureSensor}
                    \end{itemize}
            \end{itemize}

            \begin{figure}[H]
                \centering
                % \includegraphics[scale = 0.25]{Model.jpg}
                \includegraphics[scale=0.28]{Model.jpg}
                \caption{Model (Modelle)}
            \end{figure}


            \newpage
            \subsection{package \code{gson}}

                \vspace*{1cm}
                \subsubsection{class \code{Proband}}


                    \begin{enumerate}
                        \item Attribute
                            \begin{itemize}
                                \item {\large \code{probandID:String}}\\
                                    die einzige ausgeteilte ID für jeden Proband
                                \item {\large \code{studyID:String}}\\
                                    die ID der Studie
                                \item {\large \code{gender:String}}\\
                                    Geschlecht des Probands
                                \item {\large \code{age:int}}\\
                                    Alter des Probands
                                \item {\large \code{occupation:String}}\\
                                    Beruf des Probands
                                \item {\large \code{questionnaires:Collection<Questionnaire>}}\\
                                    alle zu antwortene Fragebögen
                            \end{itemize}

                        \item Methoden
                            \begin{itemize}
                                \item {\large \code{login()}}\\
                                    Nach Aufruf von Methode \code{logIn()} werden folgende Schritte durchgeführt:
                                    \begin{figure}[H]
                                        \centering
                                        \includegraphics[scale = 0.5]{logIn.jpg}
                                        \caption{\code{logIn()}}
                                    \end{figure}

                                \item {\large \code{submitAnswer(QuestionnaireID:String)}}\\
                                    Nach Klicken von Button ``Submit'' wird diese Methode aufgerufen. Die Antwort von Fragebogen mit \code{QuestionnaireID} wird zuerst lokal gespeichert. Dann versucht unsere Android-App, die Antwort auf den Server hochzuladen.
                                \item {\large \code{chooseQuestionnaire(QuestionnaireID:String):Questionnaire}}\\
                                    Mit \code{QuestionnaireID} kann der Proband dem bestimmten Fragebogen auswählen.
                                \item {\large \code{answerQuestionniare(QuestionnaireID:String)}}\\
                                    Mit \code{QuestionnaireID} kann der Proband dem bestimmten Fragebogen beantworten.
                                \item {\large \code{answerQuestion(QuestionnaireID:String, QuestionID:String)}}\\
                                    Durch Aufruf dieser Methode kann der Proband die bestimmte Frage, deren ID \code{QuestionID} ist, von Questionnaire \code{QuestionnaireID} beantworten.
                                \item {\large setters und getters für alle Attribute}
                            \end{itemize}
                    \end{enumerate}




                \subsubsection{class \code{Questionnaire}}

                    \begin{enumerate}
                        \item Attribute
                            \begin{itemize}
                                \item {\large \code{questions:Collection<Question>}}\\
                                    enthält alle Fragen dieses Fragebogens
                                \item {\large \code{isValid:boolean}}\\
                                    Falls der Proband innerhalb des gültigen Zeitbereich den Fragebogen nicht beantworten, wird den Wert dieser Attribute als ``false'' gesetzt.
                                \item {\large \code{beginTime:String}}
                                    Anfangszeitpunkt des Fragebogens
                                \item {\large \code{duration:int}}
                                    Dauer des gültigen Zeitbereich
                                \item {\large \code{QuestionnaireID:String}}
                                    die einzige ID des Fragebogens
                                \item {\large \code{shouldTrigger:boolean}}
                                    ob der Fragebogen ausgelöst werden soll
                                \item {\large \code{sensorValues:boolean[]}}
                                    \begin{itemize}
                                        \item Länge = Anzahl die von Handy angebotenen Sensoren
                                        \item Jeder Sensor entpricht einem Element in dieser Array.
                                        \item Falls der Wert eines Sensors die von Studiensleiter gesetzte Kriterien erfüllt, wird der Wert des entsprechenden Element ``true'' gesetzt. Sonst ``false''.
                                        \item z.B. Die Handy bietet 4 Sensoren an: Accelerometer-Sensor, MagneticFeld-Sensor, Temperature-Sensor und Gravity-Sensor. Die Länge dieser Array ist 4. Das erste Array-Element entspricht Accelerometer-Sensor, die zweite entspricht MagneticFeld-Sensor,usw. Jetzt erfüllen die Werte von Accelerometer-Sensor und MagneticFeld-Sensor die gesetzte Kriterien. Daher sieht \code{sensorValues} so aus: \\
                                        \code{sensorValues = {true, true, false, false}}


                                    \end{itemize}
                                \item {\large \code{shouldTriggeredBySensor:booelan}} \\
                                    Ergebnis der ``and'' Operation mit allen Elemente in \code{sensorValues}\\
                                    z.B. sensorValues = {true, true, false, true}, dann \\
                                    \code{shouldTriggeredBySensor = true \&\& true \&\& false \&\& true = false}
                                \item {\large \code{sensorManager:SensorManager}}\\
                                    Verwaltung von allen Snesoren
                            \end{itemize}
                        \item Methoden
                            \begin{itemize}
                                \item {\large \code{saveAnswerLocally()}}
                                \item {\large \code{uploadAnswer(serverAddress:String)}}\\
                                    Hochladen von Antwort auf den Server mit Adresse \code{serverAddress}
                                \item {\large \code{notifyToAnsewr()}}\\
                                    sendet Handy-Notifikation, um den Proband zu erinnern, die ausgelöste Fragebögen zu beantworten.
                                \item {\large setters und getters für alle Attribute}
                            \end{itemize}
                    \end{enumerate}

                \subsubsection{class \code{Question}}

                    \begin{enumerate}
                        \item Attribute
                            \begin{itemize}
                                \item {\large \code{question:String}}\\
                                    Inhalt der Frage
                                \item {\large \code{questionType: QuestionStrategy}}\\
                                    Art dieser Frage
                                \item {\large \code{questionID: String}}\\
                                    die einzige ID dieser Frage
                                \item {\large \code{nextQuestionID: String}}
                            \end{itemize}
                        \item Methoden
                            \begin{itemize}
                                \item {\large setters und getters für alle Attribute}
                            \end{itemize}
                    \end{enumerate}

                \subsubsection{abstract class \code{QuestionStrategy<T>}} %super abstract class

                    Hier wird Entwurfsmuster Stagtegie (eng: Strategy) eingesetzt, indem \code{QuestionStrategy} als Teil der Aggregation von \code{Question} und gleichzeitig als Oberklasse (eng: super class) von \code{SingleChoice}, \code{MutipleChoice}, \code{StepScale}, \code{DragScale} und \code{TextAnswer} gesetzt wird.
                    \begin{enumerate}
                        \item Attribute
                            \begin{itemize}
                                \item {\large \code{answer:T}}\\
                                    Generische Programmierung (eng:Generic) wird für attribute \code{answer} angewandet, da der Datentyp von Attribute \code{answer} in vershiedener Unterlasse von \code{QuestionStrategy} unterschiedlich sein sollte.
                                % \item \code{}
                                % \item \code{}
                            \end{itemize}
                        \item Methoden
                            \begin{itemize}
                                \item {\large setters und getters für alle Attribute}
                            \end{itemize}
                        \item Unterklasse\\
                            Laut Pflichtenheft gibt es fünf Fragearten. Deswegen gibt es fünf Unterklassen:
                            \begin{itemize}
                                \item {\large class \code{SingleChoice}}
                                \item {\large class \code{MultipleChoice}}
                                \item {\large class \code{StepScale}}
                                \item {\large class \code{DragScale}}
                                \item {\large class \code{TextAnswer}}
                            \end{itemize}
                    \end{enumerate}


            \subsection{package \code{Sensor}}
            \begin{figure}[H]
                \centering
                \includegraphics[scale = 0.5]{ClassDiagramAppSensor.jpg}
                \caption{Klassendiagramm für Sensoren in der Android Applikation }
            \end{figure}
                \subsubsection{class \code{SensorManager}}
                Diese Klasse ist einer Manager aller Sensoren, die in diesem Fragebogen bunutzt werden.
                \begin{enumerate}
                  \item Attribute
                    \begin{itemize}
                      \item {\large\code{sensors:Collection<Sensor>}}\\
                      enthält alle Sensoren, die in diesem Fragebogen bunutzt werden.
                    \end{itemize}
                \end{enumerate}
                \subsubsection{class \code{AccelerometerSensor}}
                Diese Klasse handelt sich um den Sensor über Beschleunigung.
                \subsubsection{class \code{GravitySensor}}
                Diese Klasse handelt sich um den Sensor über Gravitation.
                \subsubsection{class \code{MagneticFeldSensor}}
                Diese Klasse handelt sich um den Sensor über Magnetfeld.
                \subsubsection{class \code{TemperatureSensor}}
                Diese Klasse handelt sich um den Sensor über Temperatur

        \newpage
        \section{View}

            Dieser Paket wird automatisch erstellt,  wenn man mit Android Studio ein neues Program erstellt.

            \vspace*{1cm}
            \begin{figure}[H]
                \centering
                \includegraphics[scale = 0.7]{view.jpg}
                \caption{view (Präsentation)}
            \end{figure}

            \begin{itemize}
                \item {\large layout} \\
                    enthält layout-Datei von \code{Activity} und \code{Fragment}
                \item {\large mipmap} \\
                    enthält die in unsere Android-App verwendete Bilder.
                \item {\large values}\\
                    enthält lokale Ressource für Präsentation (eng: view resource) unserer Android-App.
                \item {\large drawable}\\
                    enthält die in unsere Android-App verwendete Bilder.
            \end{itemize}




        \section{Control}
            Der Teil Control besteht aus zwei Pakete: activity und fragment.

            \vspace*{1cm}
            \begin{figure}[H]
                % \centering
                \includegraphics[scale = 0.35]{Control.jpg}
                \caption{Klassendiagramm für Control in der Android Applikation }
            \end{figure}

            \subsection{activity}

                \subsubsection{class \code{ActivityManager}}
                Diese Klasse ist ein Manager aller Aktivitäten. Durch diese Klasse kann ein Benutzer dieser Applikation die auf System laufende Aktivitäten verwalten. Z.B. die Informationen einer Aktivität erhalten.
                \begin{enumerate}
                \item Attribute
                     \begin{itemize}
                            \item {\large\code{activities: Collection<Activity>}}\\
                            enthält alle Aktivitäten auf System.
                     \end{itemize}
                \item Methoden
                     \begin{itemize}
                                \item {\large addActivity()}\\
                                füget eine Aktivität den Liste hinzu.
                                \item {\large removeActivity()}\\
                                entfernt eine Aktivität aus den Liste.
                                \item {\large finishAll()}\\
                                alle Aktivitäten in den Liste abschließen.
                     \end{itemize}
                \end{enumerate}
                \subsubsection{class \code{BaseActivity}}
                \begin{enumerate}
                        \item Attribute
                        \item Methoden
                            \begin{itemize}
                                \item {\large onCreate()}\\
                                Wenn eine Aktivität erste erstellt wird, wird diese Methode angerufen.
                                \item {\large onStart()}\\
                                Wenn eine Aktivität zur Kunde sichtbar wird, wird diese Methode angerufen.
                                \item {\large onResume()}\\
                                Wenn eine Aktivität mit den Kunde  interagiert, wird diese Methode angerufen.
                                \item {\large onPause()}\\
                                Wenn das System eine vorherige Aktivität erstellt, wird diese Methode angerufen.
                                \item {\large onDestroy()}\\
                                Vor die Zerstörung einer Aktivität wird diese Methode angerufen.
                            \end{itemize}
                        \item Unterklassen
                            \begin{itemize}
                                \item {\large class \code{LogInActivity}}\\
                                Anmelden-Aktivität.
                                \item {\large class \code{AnswerQuestionActivity}}\\
                                Fragebogensbeantworten-Aktivität.
                            \end{itemize}

                \end{enumerate}
                % \subsubsection{class \code{LogInActivity}}
                % \begin{enumerate}
                %         \item Attribute
                %         \item Methoden
                % \end{enumerate}
                % \subsubsection{class \code{AnswerQuestionActivity}}
                %  \begin{enumerate}
                %         \item Attribute
                %         \item Methoden
                % \end{enumerate}

            \subsection{fragment}

                \subsubsection{class \code{ChooseQuestionnaireFragment}}
                \begin{enumerate}
                \item Attribute
                \item Methoden
                     \begin{itemize}
                                \item {\large queryQuestionnaire()}
                     \end{itemize}
                \end{enumerate}

                \subsubsection{class \code{QuestionFragment}}
                 \begin{enumerate}
                        \item Attribute
                        \item Methoden
                \end{enumerate}


        \newpage
        \section{Hilfpaket}


            \subsection{package \code{service}}

                Die Klassen in diesem Paket handeln sich um die Hintergrunddienste für unsere Android-App. Die Aufgaben, wie z.B., Hochladen von Antworten, sollen im Hintergrund erledigen, falls die Internetverbindung verfügbar ist, während der Nutzer gerade andere Applikation nutzt oder sein Gerät garnicht in der Hand hat.

                \vspace*{0.5cm}
                \begin{figure}[H]
                    \centering
                    \includegraphics[scale = 0.5]{service.jpg}
                    \caption{package \code{service}}
                \end{figure}

                \subsubsection{AutoDownloadService}
                    \begin{enumerate}
                        \item Attribute
                        \item Methoden
                            \begin{itemize}
                                \item {\large \code{static downloadQuestionnaires(serverAddress:String): List<Questionnaire>}}\\
<<<<<<< HEAD

=======
                                Nach Aufruf von dieser Methode sendet unsere Android-App dem Server Anfrage und lädet alle von studiensleiter gesetzte Fragebögen in Form von JSON-Datei herunter.
>>>>>>> 09dce143b6b2f82b8d69f5de2137325ba3bcc169
                            \end{itemize}
                    \end{enumerate}

                \subsubsection{AutoUploadService}
                    \begin{enumerate}
                        \item Attribute
                        \item Methoden
                            \begin{itemize}
                                \item {\large \code{static uploadAnswer(serverAddress:String)}}\\
                                    lädt die Antwort in Form von JSON-Datei auf den Server hoch
                            \end{itemize}
                    \end{enumerate}

            \newpage
            \subsection{package \code{util}}

                \vspace*{3cm}
                \begin{figure}[H]
                    \centering
                    \includegraphics[scale = 0.5]{util.jpg}
                    \caption{package \code{util}}
                \end{figure}

                \subsubsection{class \code{HttpUtil}}

                    Diese Klasse ist zuständig für die Kommunikation zwischen Android-App und Server, indem wir die http-Framework OkHttp\footnote{http://square.github.io/okhttp/} verwenden.

                    \begin{enumerate}
                        \item Attribute
                        \item Methoden
                            \begin{itemize}
                                \item {\large \code{static sendOkHttpRequest(address:String, callback:okhttp3.Callback)}}
                                    sendet dem Server Anfrage mittels OkHttp
                            \end{itemize}
                    \end{enumerate}

                \subsubsection{class \code{Util}}

                    Diese Klasse bietet einige zweckmäßige Funktionen an, z.B. Analysierung der  von Server zurückgegebene Daten.

                    \begin{enumerate}
                        \item Attribute
                        \item Methoden
                            \begin{itemize}
                                \item {\large \code{static handleQuestionnaireResponse(response:String):Questionnaire}}\\
                                Diese Method analysiert die von Server zurückgegebene JSON-Daten und gibt \code{Questionnaire} Objekt zurück.
                            \end{itemize}
                    \end{enumerate}
<<<<<<< HEAD
    }

=======
        }
        
            
           
>>>>>>> 09dce143b6b2f82b8d69f5de2137325ba3bcc169
    \chapter{Interface zwishcen Server und App}
        Die App- und Server-Seiten kommunizieren miteinander per Internet durch das HTTP-Protokoll. Zum Transfer von Daten wird JSON (JavaScript Object Notation) als das Datenformat genutzt.

        \section{Vom Server nach der App}
            \subsection{Anfrage um Proband-Daten}
                Nur die Daten, die vom Studienleiter bestimmt werden, werden an die Probanden gefragt. Eine JSON-Datei als Beispiel:
                \begin{lstlisting}
{
    "birthdayRequested": "true",
    "occupationRequested": "false",
    "genderRequested": "true"
}
                \end{lstlisting}

            \subsection{Übertragung der Fragebogen vom Server nach der App}
                Eine JSON-Datei als Beispiel:
                \begin{lstlisting}
{
    "study": {
        "id": 5,
        "name": "study name",
        "beginningDate": {
        "year": 2017,
        "month": 1,
        "day": 1
    },
        "endDate": {
        "year": 2017,
        "month": 7,
        "day": 1
    },
        "questionnaires": [
            {
                "id": 1,
                "name": "question name",
                "triggerEvent": {},
                "answeringTime": {"hours": 5},
                "questions": [
                    {
                        "question1": {
                              "type": "choiceQuestion",
                              "options": ["option1","option2"],
                              "followUpQuestion": 2,
                              "followUpQuestionTriggerOptions": []
                        }
                    }
                ],
                "answers": [
                    {
                        "answer1": {
                              "type": "multiChoiceQuestionAnswer",
                              "content": [1,2]
                        }
                    }
                ]
            },
            {
                "id": 1,
                "name": "question name",
                "triggerEvent": {},
                "answeringTime": {"hours": 5},
                "questions": [],
                "answers": []
            }
        ]
    }
}
            \end{lstlisting}



    \glsaddall
    \printglossary

    % Abbildungsverzeichnis
    \listoffigures

\end{document}
