\documentclass[a4paper]{scrreprt}

\usepackage[german]{babel}
\usepackage[utf8]{inputenc}
\usepackage[T1]{fontenc}
\usepackage{ae}
\usepackage[bookmarks,bookmarksnumbered]{hyperref}
\usepackage{graphicx}
\usepackage{float}
\usepackage{color}
\usepackage{color,soul}
\setstcolor{red}

\usepackage{pdfpages}
\usepackage{soul}
\graphicspath{ {Images/} }
\setcounter{secnumdepth}{5}

\begin{document}

    \begin{flushright}
        \includegraphics[scale = 0.7]{kit-logo.jpg}\\[0.5cm]
    \end{flushright}
    \vspace*{2cm}

    \begin{center} \large

        Praxis der Softwareentwicklung
        \vspace * {1.5cm}

        \textbf{\huge Mind Rate}

        \vspace*{1cm}


        {\Large Ein interaktives System mit Android-Client f\"ur Studien nach Experience-Sampling-Method (ESM)}

        \vspace*{1cm}

        \textbf{\Large Qualit\"atssicherung}
        \vspace*{2cm}

        Shanshan Du, Yi Ge, Renhan Lou, Ruoheng Ma, Haobin Tan
        \vspace*{1cm}

        26. M\"arz 2017
        \vspace*{2.5cm}

        Betreuung: Anja Exler, Dr. Andrea Schankin\\[0.5cm]
        Forschungsgruppe TECO: Technology for Pervasive Computing\\[0.5cm]

        Karlsruher Institut für Technologie
    \end{center}
    \thispagestyle{empty}

    \tableofcontents

    \chapter{Einleitung}

        Nach der Implementierungsphase erhalten wir die funktionsf\"ahige Software, die in ein Web-Interface und eine Android-App unterteilt. In der Qualit\"atssicherungsphase testen wir die Software sowohl weiter in dieser zwei Teilen als auch als eine Einheit. Die Teste bestehen aus Integrationstest und Systemtest. Dabei nehmen wir die (modifizierten) globalen Testf\"alle und Testszenarien aus dem Pflichtenheft und f\"uhren sie automatisch durch Unit-test und auch manuell durch.



			

      \newpage
      \chapter{Integrationstest}
      Folgende Testf\"alle wurden durchgef\"uhrt:


	      \section{Web-Interface}
		      \begin{itemize}
		      		          \item \textbf{/TF10/ Registrieren des Studienleiters}
		      		          \begin{enumerate}
		      		               \item \par \textbf{Stand: }Offene Homepage des Web-Interface
		      		                \par \textbf{Aktion: }Studienleiter klickt auf den Button ``Sign up''
		      		                \par \textbf{Reaktion: }Der Browser wird zu der Registrieren-Seite umgeleitet
		      			            \item \par \textbf{Stand: }Studienleiter ist in der Registrieren-Seite
		      		                \par \textbf{Aktion: }Studienleiter gibt einen Username, eine Email-Adresse und ein Passwort ein
		      		                \par \textbf{Reaktion: }Eine Bestätigungsmail wird nach dieser Email-Adresse geschickt
		      		           \end{enumerate}
		      		           \vspace*{0.3cm}
		      		           \par \textbf{Ergebnis: }\textcolor{green}{Erfolg}
		      		           \vspace*{0.6cm}
		      		           
		                          \item \textbf{/TF20/ Anmelden des Studienleiters}
		                              \begin{enumerate}
		                                  \item \par \textbf{Stand: }Offene Homepage des Web-Interface
		                                      \par \textbf{Aktion: }Registrierte Username, Email-Adresse und Passwort eingeben, und auf “Sign in” klicken
		                                      \par \textbf{Reaktion: }Erfolgreich eingeloggt. Der Browser wird zum Dashboard umgeleitet.
		                              \end{enumerate}		         
		      		           \vspace*{0.3cm}
		      		           \par \textbf{Ergebnis: }\textcolor{green}{Erfolg}
		      		           \vspace*{0.6cm}  
		      		       
		                      \item /TF30/ Vergessenes Passwort neu setzen \textbf{(Entfernt)}
		      		        \vspace*{0.6cm}
                              
                            \item \textbf{/TF30/ Vergessenes Passwort neu setzen}
                                \begin{enumerate}
                                    \item \par \textbf{Stand: }Offene Homepage des Web-Interface
                                        \par \textbf{Aktion: }Studienleiter klickt auf den Button ``Forgot password''
                                        \par \textbf{Reaktion: }Der Browser wird zu der Password-Recovery-Seite umgeleitet
                                    \item \par \textbf{Stand: }Studienleiter ist in der Password-Recovery-Seite
                                        \par \textbf{Aktion: }Studienleiter gibt den Username ein
                                        \par \textbf{Reaktion: }Eine Password-Recovery-Mail wird nach dieser Email-Adresse geschickt
                                    \item \par \textbf{Stand: }Studienleiter bekommt das Password-Recovery-Mail
                                        \par \textbf{Aktion: }Studienleiter klickt den Link im Password-Recovery-Mail
                                        \par \textbf{Reaktion: }Studienleiter ist in der Password-Reset-Seite
                                    \item \par \textbf{Stand: }Studienleiter ist in der Password-Reset-Seite
                                        \par \textbf{Aktion: }Studienleiter gibt ein neues Passwort ein
                                        \par \textbf{Reaktion: }Das Passwort wird neugesetzen
                                \end{enumerate}
                                \vspace*{0.3cm}
                                \par \textbf{Ergebnis: }\textcolor{green}{Erfolg}
                                \vspace*{0.6cm}
		      		        
		      				\item \textbf{/TF40/ Ver\"andern des Passwortes}
		                              \begin{enumerate}
		                                  \item \par \textbf{Stand: }Offene Homepage des Web-Interface
		                                      \par \textbf{Aktion: }Auf eigenen Username klicken und ``Change password'' w\"ahlen
		                                      \par \textbf{Reaktion: }Der Browser wird zur Seite ``Password change'' umgeleitet
		      							\item \par \textbf{Stand: }Offene Seite ``Password change''
		                                      \par \textbf{Aktion: }Altes und neues Passwort eingeben und auf ``change my password'' klicken
		                                      \par \textbf{Reaktion: }Das Passwort wird ge\"andert                                
		                              \end{enumerate}		         
		      		           \vspace*{0.3cm}
		      		           \par \textbf{Ergebnis: }\textcolor{green}{Erfolg}
		      		           \vspace*{0.6cm}		        
		      		           
		                      \item \textbf{/TF50/ Verwaltung der Studien}
		                          \begin{enumerate}
		                              \item \par \textbf{Stand: }Offene Seite ``Site administration''
		                                    \par \textbf{Aktion: }Auf der Zeile ``Studies'' auf ``add'' klicken
		                                    \par \textbf{Reaktion: }Der Browser wird zur Seite ``Add study'' umgeleitet
		                              \item \par \textbf{Stand: }Offene Seite ``Add study''
		                                    \par \textbf{Aktion: }Studienname, Startzeit und Endzeit eingeben, und auf ``Save'' klicken
		                                    \par \textbf{Reaktion: }Der Browser wird zur Seite ``Studies'' umgeleitet und zeigt die gerade erstellte Studie
		                              \item \par \textbf{Stand: }Offene Seite ``Studies''
		                                    \par \textbf{Aktion: }Auf den Name einer Studie klicken
		                                    \par \textbf{Reaktion: }Der Browser wird zur Seite ``Change study'' umgeleitet  
		                              \item \par \textbf{Stand: }Offene Seite ``Change study''
		                                    \par \textbf{Aktion: }Studienname, Startzeit und Endzeit ver\"andern, und auf ``Save'' klicken
		                                    \par \textbf{Reaktion: }Der Browser wird zur Seite ``Studies'' umgeleitet und zeigt die ver\"anderte Studie 
		                              \item \par \textbf{Stand: }Offene Seite ``Studies''
		                                    \par \textbf{Aktion: }Auf den K\"astchen neben den Name einer Studie klicken, ``Delete selected studies'' w\"ahlen und auf ``Go'' klicken
		                                    \par \textbf{Reaktion: }Eine Seite ``Are you sure?'' erscheint 
		      						\item \par \textbf{Stand: }Offene Seite ``Are you sure?''
		                                    \par \textbf{Aktion: }Auf ``Yes, I'm sure'' klicken
		                                    \par \textbf{Reaktion: }Die gew\"ahlte Studie wird gel\"oscht                                                                                                  
		                          \end{enumerate}
		      					\vspace*{0.3cm}
		      		           \par \textbf{Ergebnis: }\textcolor{green}{Erfolg}
		      		           \vspace*{0.6cm}  
		      		           
		                      \item \textbf{/TF60/ Verwaltung der Proband-Info-Frageb\"ogen}
		                          \begin{enumerate}
		                              \item \par \textbf{Stand: }Offene Seite ``Change study'' oder ``Add study''
		                                    \par \textbf{Aktion: }Auf ``Add another Proband Info Questionnaire'' klicken
		                                    \par \textbf{Reaktion: }Ein Feld f\"ur Erstellung des Proband-Info-Fragebogens erscheint
		                              \item \par \textbf{Stand: }Offenes Feld f\"ur Erstellung des Proband-Info-Fragebogens
		                                    \par \textbf{Aktion: }Einen oder mehreren aus ``ask for birthday'', ``ask for occupation'' und ``ask for gender'' w\"ahlen und verschiedene Sorten von Fragen erstellen. Dann auf ``Save'' klicken
		                                    \par \textbf{Reaktion: }Der Browser wird zur Seite ``Studies'' umgeleitet   
		                              \item \par \textbf{Stand: }Offene Seite ``Change study''
		                                    \par \textbf{Aktion: }Auf bereits existierendes ``proband info questionnaire'' klicken
		                                    \par \textbf{Reaktion: }Ein Feld f\"ur Ver\"andern des Proband-Info-Fragebogen erscheint  
		                              \item \par \textbf{Stand: }Offenes Feld f\"ur Ver\"andern des Proband-Info-Fragebogens
		                                    \par \textbf{Aktion: }Fragebogen ver\"andern und auf ``Save'' klicken
		                                    \par \textbf{Reaktion: }Der Browser wird zur Seite ``Studies'' umgeleitet               
		                              \item \par \textbf{Stand: }Offene Seite ``Change study'' oder ``Add study''
		                                    \par \textbf{Aktion: }Auf ``-'' bei ``proband info questionnaire'' klicken
		                                    \par \textbf{Reaktion: }Der Proband-Info-Fragebogen wird gel\"oscht                                                                           
		                          \end{enumerate}
		      					\vspace*{0.3cm}
		      		           \par \textbf{Ergebnis: }\textcolor{green}{Erfolg}
		      		           \vspace*{0.6cm}  	
		      		           
		                      \item \textbf{/TF70/ Verwaltung der Frageb\"ogen}
		                          \begin{enumerate}
		                              \item \par \textbf{Stand: }Offene Seite ``Change study'' oder ``Add study''
		                                    \par \textbf{Aktion: }Auf ``Add another Questionnaire'' klicken
		                                    \par \textbf{Reaktion: }Ein Feld f\"ur Erstellung des Fragebogens erscheint
		      						\item \par \textbf{Stand: }Offene Seite ``Change study'' oder ``Add study''
		                                    \par \textbf{Aktion: }Auf einen bereits existierenden Fragebogen klicken
		                                    \par \textbf{Reaktion: }Ein Feld f\"ur Ver\"andern des Fragebogens erscheint                              
		                              \item \par \textbf{Stand: }Offenes Feld f\"ur Erstellung oder Ver\"andern des Fragebogens
		                                    \par \textbf{Aktion: }``Questionnaire name'', ``Valid time duration after triggered'' und `Maximal trigger times per day'' eingeben und auf ``Save'' klicken
		                                    \par \textbf{Reaktion: }Der Browser wird zur Seite ``Studies'' umgeleitet   
		                              \item \par \textbf{Stand: }Offenes Feld f\"ur Erstellung oder Ver\"andern des Fragebogens
		                                    \par \textbf{Aktion: }Auf ``trigger event'' klicken
		                                    \par \textbf{Reaktion: }Ein Feld f\"ur Erstellung oder Ver\"andern der ausl\"osenden Ereignissen erscheint  
		                              \item \par \textbf{Stand: }Offenes Feld f\"ur Erstellung oder Ver\"andern der ausl\"osenden Ereignissen
		                                    \par \textbf{Aktion: }Zugeh\"orige Optionen w\"ahlen, Werte eingeben und auf ``Save'' klicken
		                                    \par \textbf{Reaktion: }Ausl\"osende Ereignisse werden erstellt und der Browser wird zur Seite ``Studies'' umgeleitet    
		                              \item \par \textbf{Stand: }Offene Seite ``Change study'' oder ``Add study''
		                                    \par \textbf{Aktion: }Auf ``-'' bei einem Fragebogen klicken
		                                    \par \textbf{Reaktion: }Der Fragebogen wird gel\"oscht                                                                                                                                     
		                          \end{enumerate}
		      					\vspace*{0.3cm}
		      		           \par \textbf{Ergebnis: }\textcolor{green}{Erfolg}
		      		           \vspace*{0.6cm}                  
		      
		                      \item \textbf{/TF80/ Verwaltung der Fragen}
		                              \begin{enumerate}
		                              \item \par \textbf{Stand: }Offenes Feld f\"ur Erstellung oder Ver\"andern eines Fragebogens
		                                    \par \textbf{Aktion: }Auf ``Add another Text Question'', ``Add another Single Choice Question'', ``Add another Multi Choice Question'' oder ``Add another Drag Scale Question'' klicken
		                                    \par \textbf{Reaktion: }Ein Feld f\"ur Erstellung der Frage von gew\"ahltem Typ erscheint         
		                              \item \par \textbf{Stand: }Offenes Feld f\"ur Erstellung oder Ver\"andern eines Fragebogens
		                                    \par \textbf{Aktion: }Auf eine bereits existierende Frage klicken
		                                    \par \textbf{Reaktion: }Ein Feld f\"ur Ver\"andern der Frage erscheint                                                     
		                              \item \par \textbf{Stand: }Offenes Feld f\"ur Erstellung oder Ver\"andern einer Frage
		                                    \par \textbf{Aktion: }Die Position der Frage bestimmen und entscheiden, ob die Frage standardm\"aßig erscheint. Alle andere geforderten Informationen eingeben. Dann auf ``Save'' klicken
		                                    \par \textbf{Reaktion: }Die Frage wird gespeichert und der Browser wird zur Seite ``Studies'' umgeleitet    
		                              \item \par \textbf{Stand: }Offenes Feld f\"ur Erstellung oder Ver\"andern eines Fragebogens
		                                    \par \textbf{Aktion: }Auf dem Kreuz neben eine bereits existierende Frage klicken und dann auf ``Save'' klicken
		                                    \par \textbf{Reaktion: }Die Frage wird gel\"oscht und der Browser wird zur Seite ``Studies'' umgeleitet
		                          \end{enumerate}
		      					\vspace*{0.3cm}
		      		           \par \textbf{Ergebnis: }\textcolor{green}{Erfolg}
		      		           \vspace*{0.6cm} 
		      
		      
		                      \item \textbf{/TF90/ Antwortsstatus besichtigen}
		                          \begin{enumerate}
		                              \item \par \textbf{Stand: } Der Studienleiter ist angemeldet und der Browser ist auf der Seite ``Site administration''
		                                    \par \textbf{Aktion: }Auf ``Studies'' klicken
		                                    \par \textbf{Reaktion: }Der Browser wird zur Seite ``Studies'' umgeleitet und bei jeder Studie wird ``Answer updated times'' gezeigt
		                          \end{enumerate}		       
		      					\vspace*{0.3cm}
		      		           \par \textbf{Ergebnis: }\textcolor{green}{Erfolg}
		      		           \vspace*{0.6cm} 	
		      		           
		      				\item /TF65/ Feedback (Motivation) senden \textbf{(Entfernt)}	
		      				\vspace*{0.6cm}	
		      				
		                      \item \textbf{/TF100/ Exportieren von Daten}
		                      \begin{enumerate}
		                          \item \par \textbf{Stand: }Offene Seite ``Studies''
		                                \par \textbf{Aktion: }Eine Studie w\"ahlen, im unteren Kombinationsfeld ``Export csv'' w\"ahlen und auf ``Go'' klicken
		                                \par \textbf{Reaktion: }Eine CSV-Datei mit Information der Studie wird heruntergeladen
		                      \end{enumerate}	
		      					\vspace*{0.3cm}
		      		           \par \textbf{Ergebnis: }\textcolor{green}{Erfolg}
		      		           \vspace*{0.6cm}                 			           
		      		           
		      		           \end{itemize} 
	
	
	      \section{Android-App}
              \begin{itemize}
                  \item \textbf{/TF110/ Erstes Anmelden der Probanden}
                  \begin{enumerate}
                        \item \par \textbf{Stand: }Der Proband hat sich noch nicht in der App angemeldet
                        \par \textbf{Aktion: }Der Proband öffnet die App
                        \par {\color{red}{\textbf{\st{Reaktion: }}\st{Die App wechselt auf die ``User Login'' Seite}}}

                        \par \textbf{{\color{blue}{Reaktion: }}} {\color{blue}{Die App wechselt auf die ``choose Language'' Seite}}
                        \item \par \textbf{{\color{blue}{Stand: }}}{\color{blue} Die ``choose Language'' Seite liegt vor}
                        \par \textbf{{\color{blue}{Aktion: }}}{\color{blue}Der Proband wählt die Sprache aus, und klickt den Button ``OK''}
                        \par \textbf{{\color{blue}{Reaktion: }}}{\color{blue}Der Proband wird auf die ``User Login'' Seite weitergeleitet}
                        \item \par \textbf{Stand: }Die ``User Login'' Seite liegt vor
                        \par {\color{red}{\textbf{\st{Aktion: }}\st{Der Proband gibt Studie-ID und die angeforderte Information ein, und klickt den Button ``Submit''}}}
                        \par \textbf{{\color{blue}{Aktion: }}}{\color{blue}Der Proband gibt Studie-ID und Proband-ID ein,dann gibt er auch die angeforderte Information ein, und klickt den Button ``Log In''}
                        \par \textbf{Reaktion: }Der Proband wird auf die Hauptseite weitergeleitet
                  \end{enumerate}
                  \item \textbf{/TF105/ Automatisches Anmelden der Probanden}
	              \begin{enumerate}
	        	        \item \par \textbf{Stand: }Der Proband hat sich schon einmal in der App angemeldet
	        	        \par \textbf{Aktion: }Der Proband öffnet die App
	        	        \par \textbf{Reaktion: }Der Proband wird automatisch auf die Hauptseite weitergeleitet
	              \end{enumerate}
                  \item \textbf{/TF110/ Antworten eines Fragebogens}
	              \begin{enumerate}
                        \item \par \textbf{Stand: }Die Hauptseite der App liegt vor
                        \par {\color{red}{\textbf{\st{Aktion: }}\st{Der Proband w\"ahlt einen Fragebogen aus ``My Questionnaires'' }}}
                        \par {\color{blue}{\textbf{Aktion: }Der Proband klickt den Men\"u-Button und den Button`` Questionnaires'' im Men\"u}}
                        \par {\color{red}{\textbf{\st{Reaktion: }}\st{Der Proband wird auf die erste Frage des gew\"ahlten Fragebogens weitergeleitet}}}
                        \par {\color{blue}{\textbf{Reaktion: }Der Fragebogen \"uber die Informationen des Probandes ``Hello World! '' ist vorliegend. Proband kann zuerst diesen Proband-Info-Fragebogen antworten.}}
                        \item \par {\color{blue}{\textbf{Stand: }Der Proband hat diesen Proband-Info-Fragebogen beantwortet und ein Fragebogen ist ausgel\"ost. }}
                        \par {\color{blue}{\textbf{Aktion: }Der Proband w\"ahlt diesen ausgelösten Fragebogen aus ``Questionnaires'' }}
                        \par {\color{blue}{\textbf{Reaktion: }Der Proband wird auf die erste Frage des gew\"ahlten Fragebogens weitergeleitet}}

	        	        \item \par \textbf{Stand: }Der Proband ist auf der Seite einer Frage
	        	        \par \textbf{Aktion: }Der Proband beantwortet die Frage und klickt den Button ``Next Question''
	        	        \par \textbf{Reaktion: }Der Proband wird auf die nächste Frage des Fragebogens weitergeleitet
	        	        \item \par \textbf{Stand: }Der Proband ist auf der Seite einer Frage
	        	        \par \textbf{Aktion: }Der Proband klickt den Button ``Homepage''
	        	        \par \textbf{Reaktion: }Der Proband wird auf die Hauptseite weitergeleitet und die schon eingegebenen Antworten des Fragebogens werden nicht gespeichert
	        	        \item \par \textbf{Stand: }Der Proband ist auf der Seite letzter Frage des Fragebogens
	        	        \par \textbf{Aktion: }Der Proband beantwortet die Frage und klickt den Button ``Submit''
	        	        \par \textbf{Reaktion: }Die App zeigt ``You have successfully submitted your answers!'' und wechselt auf die Hauptseite
	              \end{enumerate}

              \end{itemize}
	
	
	  \newpage
	  \chapter{Systemtest}
	  Folgende Testszenarien wurden durchgef\"uhrt:
	

		\section{Szenario 1}
                \par Ein super psychologischer Expert Prof. Dr. Mata beschließt, eine Studie nach ESM Methode durchzuf\"uhren. Er \"offnet den Browser und die Mind-Rate Web-Anwendung. Nach Eingeben g\"ultiger Email-Adresse, Username und Passwort gelingt ihm zu registrieren und erh\"alt er dann eine Bestätigungsmail.
                
                \par Nach Anmeldung setzt Prof. Dr. Mata zun\"achst den Name und Zeitraum seiner Studie. Er entscheidet sich, die Studie ``Hello World'' zu nennen und die Studie wird 2 Monate dauern. Dann setze er einen neuen Proband-Info-Fragebogen, um Informationen von seinen Probanden anzufordern. Er w\"ahlt ``ask for birthday'' und ``ask for gender''.
                
                \par Danach beginnt Prof. Dr. Mata, einen neuen Fragebogen zu erstellen. Er nennt den Fragebogen ``9527'' und setzt, dass er maximal 10 mal am Tag ausgel\"ost werden kann. Nach Erscheinung hat der Proband 1 Stunde den Fragebogen zu beantworten. Dann w\"ahlt Prof. Dr. Mata SMS als ausl\"osendes Ereignis des Fragebogens.
                
                \par Jetzt ist Prof. Dr. Mata in der Lage, die Fragen einzugeben und Art der Fragen auszuw\"ahlen. Vier Fragenarten sind verf\"ugbar: ``Text Question'', ``Single Choice Question'', ``Multi Choice Question'' und ``Drag Scale Question''.
                
                \par Prof. Dr. Mata schreibt die erste Frage ``Where are you?'' und setzt sie als Single-Choice-Frage. Für Antwort stellt er 4 Optionen zur Verf\"ugung: ``Home'', ``Work'', ``Travel'', ``On the way''. Dann klickt er auf den Button ``+'' neben ``Add another text question'' und f\"ugt er die zweite Frage ``Where are you heading to?'' hinzu. Danach beschließt er, die dritte Frage als Drag-Scale-Frage zu definieren. Er schreibt die Frage ``How do you feel?'' und erstellt die Stufen 1 bis 5. Zum Schluss klickt er auf ``Save'' und die Seite ``Studies'' wird gezeigt.
                
                \par Einige Minuten sp\"ater findet Prof. Dr. Mata, dass die Dauer der Studie, 2 Monate,  zu lang ist. Daher klickt er den Name der Studie ``Hello World'' und er ist zu der Seite ``Change study'' geleitet. Nachdem er die \"Anderung erledigt, klickt er auf den Button ``Save'' und er ist wieder zu der Seite ``Studies'' geleitet.
                
                \par Prof. Dr. Mata denkt wieder, dass es sinnlos ist, die zweite Frage zu beantworten, falls die Probanden Optionen ``Home'' und ``Work'' f\"ur die erste Frage gew\"ahlt haben. Deswegen ruft er die Seite f\"ur die erste Frage auf und gibt er ``3'' f\"ur'``Position of the follow up question''-Setzung unter Optionen ``Home'' und ``Work'' ein.
                
                \par Schließlich hat Prof. Dr. Mata diesen Fragebogen fertig erstellt. Er klickt dann auf den Button ``Save''. Danach macht er eine Pause. Nach der Pause macht er weiter seine Arbeit... 
		      		           \vspace*{0.3cm}
		      		           \par \textbf{Ergebnis: }\textcolor{green}{Erfolg}
		      		           \vspace*{0.6cm}                  
                

            \section{Szenario 2}
	            \par Informatikstudentin Elsa nimmt an einer ESM-Studie teil. Dafür bekommt sie eine Studien-ID ``9527'' von dem Studienleiter Prof. Dr. Mata. Sie ladet die Android-App Mind-Rate herunter und \"offnet sie. Dann wird Elsa auf der ``User Login'' Seite angefordert, die Studien-ID und ihres Alter und Geschlecht einzugeben. Nach der Eingabe wechselt die App auf ``Homepage'' und noch keinen Fragebogen wird gezeigt.

	            \par In der Nacht regnet es draussen und Elsa bekommt auf ihrem Handy eine Notifikation zum neuen Fragebogen. Sie \"offnet erneut die App und wird automatisch auf die Seite ``Homepage'' weitergeleitet. Jetzt gibt es einen Fragebogen ``Questionnaire 01: Your feelings in a raining night'' in ``My Questionnaires''. Die Antwortzeit des Fragebogens wird auf 2 Stunden begrenzt. Sie klickt auf den Fragebogen und die App wechselt auf die Seite der ersten Frage ``Where are you now?''. Sie w\"ahlt die Antwort ``At home'' aus und klickt auf ``Next Question''. Dann wird die Zweite Frage ``How do you feel at the moment?'' gezeigt. Diese ist die letzte Frage des Fragebogens. Elsa beantwortet sie mit der Stufe ``good'' und klickt auf den Button ``Submit''. Die App zeigt “You have successfully submitted your answers!” und wechselt auf die Hauptseite.

            \section{Szenario 3}
                \par Prof. Dr. Mata hat zuf\"allig mitbekommen, dass seine Studentin Elsa gerade einen Fragebogen seiner Studie ``Hello World'' beantwortet hat. Er ist so gespannt auf ihre Antwort und \"offnet die Mind-Rate Webseite. Nachdem er sich angemeldet hat, klickt er auf ``Studies'' und es gibt schon ganz viele Studien. 
                
                \par Um die Studie ``Hello World'', die er als erste erstellt hat, zu finden, klickt er in der erster Zeile ``ID''. Dann erscheint in diesem K\"astchen ein Pfeil nach oben und die Studien werden nach ihren IDs aufsteigend sortiert. Jetzt steht ganz oben die Studie ``Hello World''. In dieser Zeile sieht Prof. Dr. Mata, dass ``Answer updated times'' nun ``1'' ist.

				\par Er m\"ochte jetzt die Antwort sehen. Daher klickt er auf den K\"astchen neben den Name der Studie, w\"ahlt ``Delete selected studies'' und klickt auf ``Go''. Dann wird eine CSV-Datei heruntergeladen. Er \"offnet die Datei und kann schließlich alle Informationen \"uber die Antwort sehen: Studien-ID, Fragebogen-ID, Frage, Antwort, Antwortzeit und Sensorwert. Er ist mit seiner Arbeit sehr zufrieden.
		      		           \vspace*{0.3cm}
		      		           \par \textbf{Ergebnis: }\textcolor{green}{Erfolg}
		      		           \vspace*{0.6cm}  	


      \newpage
      \chapter{Problemmeldungen}
	      \section{Web-Interface}
		
		     \begin{itemize}
		     \item \textbf{Fehlersymptom:}
		     \par \textbf{Fehlergrund:}
		     \par \textbf{Fehlerbehebung:}
		     \end{itemize}
		
		  \section{Android-App}
		
			  \begin{itemize}
				  \item \textbf{Fehlersymptom:}
				  	\par \textbf{Fehlergrund:}
			  		\par \textbf{Fehlerbehebung:}
			  \end{itemize}
			
			
		\newpage
		\chapter{Test\"uberdeckung}
			\section{Web-Interface}
			
			
			\section{Android-App}



\end{document}
