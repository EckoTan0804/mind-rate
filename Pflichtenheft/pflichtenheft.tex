\documentclass[a4paper]{scrreprt}

\usepackage[german]{babel}
\usepackage[utf8]{inputenc}
\usepackage[T1]{fontenc}
\usepackage{ae}
\usepackage[bookmarks,bookmarksnumbered]{hyperref}
\usepackage{graphicx}
\usepackage[toc]{glossaries}
\usepackage{float}
\graphicspath{ {Images/} }
\setcounter{secnumdepth}{5}
\makeglossaries

\begin{document}
    \newglossaryentry{Versuchsleiter}{name=Versuchsleiter, description={Person, unter dessen Leitung ein Versuch durchgeführt wird}}
    \newglossaryentry{Proband}{name=Proband, description={Leute, der sich an dem Versuch beteiligt}}
    \newglossaryentry{Web-Interface}{name=Web-Interface, description={Benutzeroberfläche, die man in einem Web-Browser nutzen kann}}
    \newglossaryentry{Antwortsstatus}{name=Antwortsstatus, description={Die statistischen Daten der Antworten von Probanden innerhald von G\"ultigkeitszeitbereich des Fragebogens}}

    \begin{flushright}
        \includegraphics[scale = 0.7]{kit-logo.jpg}\\[0.5cm]
        % \includegraphics[scale = 1]{teco.jpg}
    \end{flushright}
    % \includegraphics[scale = 0.5]{kit-logo.jpg} \hspace{4cm} \includegraphics[scale = 1]{teco.jpg}
    \vspace*{2cm}

    \begin{center} \large

        Praxis der Softwareentwicklung
        \vspace * {1.5cm}

        \textbf{\huge Mind Rate}

        \vspace*{1cm}

        {\Large Ein interaktives Werkzeug f\"ur Studie nach Experience-Sampling-Method (ESM)}

        \vspace*{1cm}

        \textbf{\Large Pflichtenheft}
        \vspace*{2cm}

        Shanshan Du, Yi Ge, Renhan Lou, Ruoheng Ma, Haobin Tan
        \vspace*{1cm}

        02. Dezember 2016
        \vspace*{2.5cm}


        Betreuung: Anja Exler, Dr. Andrea Schankin, Erik Pescara\\[1cm]
        Technology f\"ur Pervasive Computing\\[0.5cm]
        Karlsruher Institut für Technologie
    \end{center}
    \thispagestyle{empty}

    \tableofcontents

    \chapter{Zielbestimmung}
    % Dieses Kapitel dient der Bestimmung von Zielen und nicht für deren Verwendung
    % notwendige Funktionen.
        Die Nutzer sollen durch das Produkt in die Lage versetzt werden, ESM-Versuche durchzuführen und Feedbacks des Versuchs zu erhalten.\\

        \noindent Das Produkt besteht aus zwei Teilsysteme: ein Web-Interface für den Versuchsleiter und eine Android-Anwendung für die Probanden.


        \section{Musskriterien}
            \subsection{Web-Interface für Versuchsleiter}
            
                \subsubsection{Verwalten der Versuchen}
                    \begin{itemize}
                        \item Erstellen, \"Andern von Versuchen
                        \item Setzen, \"Andern von Versuchsnamen, Anfangs- und Endedaten
                        \item Generieren von Versuchsnummern
                        \item Senden von Nachrichten zu allen Probanden eines Versuchs
                    \end{itemize}
                
                \subsubsection{Verwalten der Frageb\"ogen}
                    \begin{itemize}
                        \item Erstellen, Ändern von Fragebögen
                        \item Erstellen, \"Andern von Fragen
                        \item Setzen, \"Andern von Erscheinungsereignissen und Abgabeterminen der Fragebögen
                    \end{itemize}

                \subsubsection{Erfassen von Ergebnisse der Untersuchung}
                    \begin{itemize}
                        \item Besichtigen von \gls{Antwortsstatus}
                        \item Exportieren von Daten
                    \end{itemize}
            \vspace*{0.5cm}

            \subsection{Android-App f\"ur Probanden}

                \subsubsection{Anmelden der Probanden durch gegebene Versuchsnummer}

                \subsubsection{Antwort auf Frageb\"ogen}
                    \begin{itemize}
                        \item Notifikationen auf dem Smartphone über die Frageb\"ogen
                        \item Beantworten von Frageb\"ogen 
                    \end{itemize}

                \subsubsection{Automatische Verwaltung von Antworten}
                    \begin{itemize}
                        \item lokales Speichern von Antworten
                        \item Protokollieren von Abgabezeit der Antworten
                        \item Hochladen von Antworten auf den Server
                    \end{itemize}

                \subsubsection{Zurgreifen einiger Sensoren von Handy}
                    \begin{itemize}
                        \item Zugreifen von Wetter
                        \item Zugreifen von GPS
                        \item Zugreifen von Aktivitätssensor
                    \end{itemize}
                \vspace*{0.5cm}


        \section{Wunschkriterien}

            \subsection{Web-Interface f\"ur Versuchsleiter}
                \begin{itemize}
                    \item Erm\"oglichen von ``angemeldet bleiben"
                    \item Erzeugen von verschiedenen (Statistik-)Diagramme f\"ur hochgeladene Daten (z.B. S\"aulendiagramm, Kreisdiagramm)
                    \item Unterst\"utzen von unterschiedlichen Sprachen

                \end{itemize}

            \subsection{Android-Anwendung f\"ur Probanden}
                \begin{itemize}
                    \item Erm\"oglichen von ``angemeldet bleiben"
                    \item Unterst\"utzen von unterschiedlichen Sprachen
                    \item Wechseln von Farbe-Themen der Anwendung
                \end{itemize}
                \vspace*{0.5cm}


        \section{Abgrenzungskriterien}
            % Abgrenzungskriterien: Diese Kriterien sollen bewusst nicht erreicht werden.
            \begin{itemize}
                \item Keine verteilte Datenbank, keine Echtzeitanforderungen, keine synchronisierten Datenbankzugriffe
                \item Keine Unterst\"utzung f\"ur iOS
            \end{itemize}

    \chapter{Produkteinsatz}
        Das Produkt dient zur Sammlung der Versuchsdaten aus ESM-Versuchen. Damit bietet sie für Versuchsleiter eine Lösung, ESM-Versuchen durchzuführen. Diese Tätigkeit soll zusätzlich im Internet und auf dem Smartphone möglich sein.

        \section{Anwendungsbereiche}
            \begin{itemize}
                \item Akademischer / Sozialwissenschaftlicher Anwendungsbereich
                \item Statistischer Anwendungsbereich
                \item Geschäftlicher Anwendungsbereich
            \end{itemize}

        \section{Zielgruppen}
            \begin{itemize}
                \item Versuchsleiter eines ESM-Versuchs
                \item Teilnehmer des Versuchs
            \end{itemize}

        \section{Betriebsbedingungen}
            \begin{itemize}
                \item Versuchsleiter: Büroumgebung
                \item Versuchsteilnehmer: im alltäglichen Leben aufs Smartphone
                \item Betriebszeit rund um die Uhr, läuft unbeaufsichtigt
            \end{itemize}

    \chapter{Produktumgebung}
        Eine Client-Server Architektur mit 2 Client-Typen: ein Web-Interface für Versuchsleiter und eine Android-App für Probanden

        \section{Software}
            \begin{itemize}
                \item Serverseite
                    \begin{itemize}
                        \item  Läuft auf Linux
                        \item Alle Softwares der Serverseite werden durch Docker verpackt
                        \item Datenbank: SQLite, verwaltet durch das Django-Framework
                        \item Programmiersprache: Python 3
                        \item Web server: Nginx
                    \end{itemize}
                \item Clientseite
                    \begin{itemize}
                        \item Web-Interface\\
                             Web-Browser, Referenzstandard Google Chrome 54
                        \item  Android-App\\
                             Android, Referenzstandard Android 5.1.1 Lollipop
                    \end{itemize}
            \end{itemize}

        \section{Hardware}
            \begin{itemize}
                \item Serverseite\\
                    Leistungsstarke Standardrechner
                \item  Web-Interface\\
                    Standardrechner (für Web-Browser)
                \item Android-App\\
                    Standardsmartphone
            \end{itemize}

    \chapter{Funktionale Anforderungen}

        \section{Web-Interface}
            \begin{itemize}
                \item \textbf{/F10/ Registrieren und Anmelden der Versuchsleiter}

                    \par \textbf{Ziel: }Registrieren oder Anmelden der Versuchsleiter in der Web-Verwaltungssystem
                    \par \textbf{Vorbedingung (Registrieren): }-keine-
                    \par \textbf{Vorbedingung (Anmelden): }Ein Konto des Leiters soll vorhanden sein.
                    \par \textbf{Nachbedingung (erfolgreiches Registrieren): }Der Leiter bekommt eine Bestätigungsmail und kann sich mit dem neu erzeugten Konto anmelden.
                    \par \textbf{Nachbedingung (erfolgreiches Anmelden): }Der Leiter ist in der Verwaltungssystem angemeldet und kann seine Versuche verwalten.
                    \par \textbf{Nachbedingung (Fehlschlag): }Der Leiter bleibt unangemeldet.
                    \par \textbf{Akteure: }Versuchsleiter
                    \par \textbf{Auslösendes Ereignis: }Der Leiter busucht die Webseite dieses Online-Verwaltungssystems.
                    \par \textbf{Beschreibung: }
                        \begin{enumerate}
                            \item Beim Registrieren soll der Leiter eine gültige Email-Adresse als Konto-Name und ein gültiges Passwort eintragen. Diese Daten werden von dem Datenbank gespeichert und der Server erzeugt ein neues Konto. Danach empfängt der Leiter eine Bestätigungsmail und kann sich mit dem neuen Konto anmelden.
                            \item Beim Anmelden soll der Leiter die registrierte Email-Adresse und sein Passwort eingeben. Wenn die Email-Adresse und das Passwort stimmen überein, dann wird der Leiter in die Verwaltungsseite weitergeleitet.
                        \end{enumerate}
                    \par \textbf{Erweiterung: }
                    \par \textbf{Alternativen: }


                \item \textbf{/F20/ Vergessendes Passwort neu setzen}

                \par \textbf{Ziel: }Versuchsleiter können vergessendes Passwort neu setzen
                \par \textbf{Vorbedingung: }Versuchsleiter hat ein Konto
                \par \textbf{Nachbedingung (Konto vorhanden): }Der Versuchsleiter bekommt ein neues zufällig generiertes Passwort.
                \par \textbf{Nachbedingung (Konto nicht vorhanden): }Eine Fehlermeldung mit den Wörtern ``Account does not exist'' erscheint.
                \par \textbf{Akteure: }Versuchsleiter
                \par \textbf{Auslösendes Ereignis: }Der Versuchsleiter klickt auf ``Reset Password''
                \par \textbf{Beschreibung: }
                \begin{enumerate}
                    \item Auf ``Reset Password'' klicken
                    \item Email-Adresse eingeben
                    \item Der Versuchsleiter bekommt ein Mail mit einem neuen zufällig generierten Passwort 
                \end{enumerate}
                \par \textbf{Erweiterung: }
                    \par Falls das Konto nicht vorhanden ist, kommt eine Fehleranzeige sofort auf diese Seite auf. Sonst sendet der Server das Mail mit dem neuen Passwort zur Mailadresse des Kontos.


                \item \textbf{/F30/ Verwaltung des Fragebogens}

                \par \textbf{Ziel: }Verwaltung des Fragebogens in einem Versuch
                \par \textbf{Vorbedingung: }Angemeldet in dem Verwaltungssystem und Mindestens Ein Versuch ist vorhanden
                \par \textbf{Nachbedingung (Erfolg): }Ein neuer Frageboden wird erstellt, oder ein vorliegender Frageboden wird wieder eingestellt.
                \par \textbf{Nachbedingung (Fehlschlag): }Ein neuer Frageboden wird nicht erstellt, oder ein vorliegender Frageboden hat keine Änderung.
                \par \textbf{Akteure: }Versuchsleiter
                \par \textbf{Auslösendes Ereignis: }Der Versuchsleiter vesucht, den ``Set questionnaire'' Button auf der Webseite zu klicken

                \par \textbf{Beschreibung: }
                \begin{enumerate}
                    \item Wenn ein benötigter Fragebogen vorhanden ist, wählt der Versuchsleiter diesen Fragebogen aus.
                    \item Wenn ein benötigter Fragebogen nicht vorhanden ist, klickt der Versuchsleiter auf den Button ``new''
                    \item Der Versuchsleiter gibt konkrete Name des Fragebogens ein
                    \item Der Versuchsleiter stellt die Anfangsbedingungen des Fragebogens ein. Z.B. die Anfangszeit und die Endeszeit. Er kann auch die benötigte Sensoren hinzufügen.
                    \item Danach liegt eine Voransicht des Fragebogens rechts auf der Webseite vor
                    \item  Der Versuchsleiter klickt auf den Button ``Refresh'', damit dir Inhalt des Fragebogens in den Voransicht gezeigt werden können
                    \item Der Versuchsleiter klickt auf den Button ``save'', damit der Fragebogen in einen Versuch gespeichert werden können
                    \item Der Versuchsleiter klickt auf den Button ``submit'', damit der Fragebogen auf den Server hochgeladen werden können
                \end{enumerate}
                \par \textbf{Erweiterung: }
                \par \textbf{Alternativen: }


                \item \textbf{/F40/ Verwaltung der Fragen }
                \par \textbf{Ziel: }Erstellung der Fragen in einem Fragebogen
                \par \textbf{Vorbedingung: } Angemeldet in dem Verwaltungssystem und Mindestens Ein Fragebogen ist vorhanden
                \par \textbf{Nachbedingung (Erfolg): }Eine neue Frage wird erstellt.
                \par \textbf{Nachbedingung (Fehlschlag): }Eine neue Frage wird nicht erstellt
                \par \textbf{Akteure: }Versuchsleiter
                \par \textbf{Auslösendes Ereignis: }Der Versuchsleiter vesucht, den ``go to question settings'' Button auf der Webseite zu klicken
                \par \textbf{Beschreibung: }
                \begin{enumerate}
                    \item Verwaltung des Fragebogen
                    \par Der Versuchsleiter wählt zuerst einen Fragebogen aus oder erstellt er einen neuen Fragebogen. Zudem gehört die neue Frage.
                    \item Erstellung und Löschen der Fragen
                    \par Der Versuchsleiter entscheidet sich die Fragenart und Antwortart. Dann soll der Versuchsleiter auch konkrete Frage eingeben.Er kann auch eine ausgewählte Frage von der Fragenliste löschen
                    \item Verwaltung der Beziehungen zwischen Fragen
                    \par Der Versuchsleiter entscheidet sich, welche Option in diesen Frage ein auslösendes Ereignis vorliegender Fragen ist.Dann gibt er Fragennummer ein.
                    \item Voransicht
                    \par Der Versuchsleiter klickt auf den Button ``Refresh'', damit die hinzugefügte Frage in den Fragebogen gezeigt werden können
                \end{enumerate}
                \par \textbf{Erweiterung: }
                \begin{enumerate}
                    \item Es gibt fünf Fragenarten. Die sind jeweils ``Single-Choice-Frage'', ``Multiple-Choice-Frage'', ``Skala mit Stufen Frage'', ``Skala ohne Stufe Frage'' und ``Offene Frage''.
                    \item Es ist nicht notwendig, die Probanden den Fragebogen sequenziell zu antworten. Der Versuchsleiter kann sich entscheiden, welche Fragen die Probanden antworten müssen und welche Fragen ein auslösende Ereignis haben.
                \end{enumerate}


                \item \textbf{(Wunsch) /F50/ \gls{Antwortsstatus} besichtigen}
                \par \textbf{Ziel: }Besichtigung der Antwortsstatus
                \par \textbf{Vorbedingung: }Der Versuchsleiter ist angemeldet.
                \par \textbf{Nachbedingung : }Der neue Antwortsstatus eines Fragebogens (z.B. Verteilung der Antwroten, Anzahl der beantworteten Probanden) ist besichtgbar.
                \par \textbf{Nachbedingung (Fehlschlag): }Der neue Antwortsstauts wird nicht gezeigt.
                \par \textbf{Akteure: }Versuchsleiter
                \par \textbf{Auslösendes Ereignis: }Der Versuchsleiter kann die \"Ubersicht von Antworten des aktuellen Fragebogens sehen.
                \par \textbf{Beschreibung: }
                \begin{enumerate}
                    \item Klicken von Button ``View answers''
                    \item Zeigen der \"Ubersicht von allen Antwort beim erfolgreichen Anmelden
                    \item Erfassen von Antwortsstatus nach Bedarf und eingegebener Kriterien
                    \item Laden des neuen Antwortsstatus beim Klicken von Button ``Refresh''
                \end{enumerate}
                \par \textbf{Erweiterung: }
                \par \textbf{Alternativen: }
                \begin{figure}[H]
                    % \raggedleft
                    \centering
                    \includegraphics[scale=0.8]{Antwortsstatus_besichtigen.jpeg}
                    \caption{Antwortsstatus besichtigen}
                \end{figure}


                \item \textbf{(Wunsch) /F55/ Nachrichten zu allen Probanden senden}
                \par \textbf{Ziel: }Senden von Nachrichten, um die Probanden zu motivieren, die Fragebögen weiter zu beantworten.
                \par \textbf{Vorbedingung: }Der Versuchsleiter ist angemeldet.
                \par \textbf{Nachbedingung : }Jeder Proband erhalten Motivation vom Versuchsleiter.
                \par \textbf{Nachbedingung (Fehlschlag): }Die Proband erhalten keine Feedbacks.
                \par \textbf{Akteure: }Versuchsleiter
                \par \textbf{Auslösendes Ereignis: }Alle Probanden erhalten die von Versuchsleiter geschriebene Motivation.
                \par \textbf{Beschreibung: }
                \begin{enumerate}
                    \item Klicken von Button ``Send Feedback / Motivation''
                    \item Schreiben von Feedback (Motivation)
                    \item Klicken von Button ``Send'', um die Feedbacks an allen Probanden zu senden.
                \end{enumerate}
                \par \textbf{Erweiterung: }
                \par \textbf{Alternativen: }


                \item \textbf{/F60/Exportieren von Daten}

                \par \textbf{Ziel: }Exportieren von Antworten und statistischen Daten
                \par \textbf{Vorbedingung: }Angemeldet in dem Verwaltungssystem
                \par \textbf{Nachbedingung (Erfolg): }Die den Versuchen zugehörigen Daten werden in CSV-Format exportiert.
                \par \textbf{Nachbedingung (Fehlschlag): }Die den Versuchen zugehörigen Daten werden nicht exportiert.
                \par \textbf{Akteure: }Versuchsleiter
                \par \textbf{Auslösendes Ereignis: }Der Leiter will die Daten exportieren.
                \par \textbf{Beschreibung: }
                \begin{itemize}
                    \item Der Leiter klickt auf "Daten Exportieren". Dann werden die zugehörigen Daten in CSV-Format exportiert.
                \end{itemize}
            \end{itemize}


    \newpage
    \section{Android-App}

        \begin{itemize}
            \item \textbf{/F70/Anmelden der Probanden}

                \par \textbf{Ziel: }Anmelden der Probanden in der App und Sammeln der Versuchsteilnehmerdaten bei dem ersten Anmelden
                \par \textbf{Vorbedingung: }-keine-
                \par \textbf{Nachbedingung (erstes Anmelden): }Versuchsteilnehmerdaten liegen vor und Proband ist angemeldet
                \par \textbf{Nachbedingung (kein erstes Anmelden): }Proband ist angemeldet
                \par \textbf{Akteure: }Proband
                \par \textbf{Auslösendes Ereignis: }Proband öffnet die App
                \par \textbf{Beschreibung: }
                \par 1. Wenn der Proband zum ersten Mal die App öffnet: Der Proband meldet sich mit der ID des Versuchs an und gibt die
                 von Versuchsleiter angeforderten Versuchsteilnehmerdaten ein. Eine einzigartige, mit dem Handy verbundene Proband-ID wird generiert und dem Proband zugeteilt. Die Proband-ID wird nicht gezeigt aber in der App gespeichert. Die App wechselt dann auf die Hauptseite.
                \par 2. Wenn der Proband sich mindestens einmal angemeldet hat: Die App wechselt automatisch auf die Hauptseite.
                \par \textbf{Alternativen: }
                \begin{figure}[ht]
                    % \raggedleft
                    \centering
                    \includegraphics[scale=1]{AppAnmelden.jpeg}
                    \caption{Anmelden der Probanden}
                \end{figure}


            \item \textbf{/F80/Beantworten eines Fragebogens}

                \par \textbf{Ziel: }Sammeln der Antworten auf den Fragebogen
                \par \textbf{Vorbedingung: }Anmeldung und mindestens ein vorliegender Fragebogen in der App
                \par \textbf{Nachbedingung: }die Antworten werden lokal gespeichert
                \par \textbf{Akteure: }Proband
                \par \textbf{Auslösendes Ereignis: }Proband erhaltet Notifikation
                \par \textbf{Beschreibung: }
                \par 1. Wenn ein zu beantwortender Fragebogen vorliegt, schickt die App eine Notifikation.
                \par 2. Nachdem der Proband sich angemeldet hat, liegt die App auf der Hauptseite mit einer Liste der auszufüllenden Fragebögen. Der Proband wählt einen Fragebogen aus und beantwortet alle darauf stehenden Fragen. Im Anschluss klickt der Proband auf ``Send'' und die Antworten werden lokal gespeichert.
                \par \textbf{Alternativen: }
                \begin{figure}[ht]
                    % \raggedleft
                    \centering
                    \includegraphics[scale=0.5]{AppAntworten.jpeg}
                    \caption{Antworten eines Fragebogens}
                \end{figure}


            \item \textbf{/F90/Versenden der Antworten eines Fragebogens}

            \par \textbf{Ziel: }Versenden der Antworten an den Server
            \par \textbf{Vorbedingung: }die Antworten werden lokal gespeichert
            \par \textbf{Nachbedingung (Erfolg): }die Antworten werden auf dem Server gespeichert
            \par \textbf{Nachbedingung (Fehlschlag): }die Antworten werden nicht auf dem Server gespeichert
            \par \textbf{Akteure: }App und Server
            \par \textbf{Auslösendes Ereignis: }die Netzwerkverbindung ist verfügbar und es gibt lokal gespeicherte Antworten
            \par \textbf{Beschreibung: }
            \par Wenn die Netzwerkverbindung vorhanden ist, schickt die App die verfügbaren Antworten an den Server und die Antworten werden auf dem Server gespeichert.
            \par \textbf{Alternativen: }
            \begin{figure}[ht]
                % \raggedleft
                \centering
                \includegraphics[scale=1.2]{AppVersenden.jpeg}
                \caption{Versenden der Antworten eines Fragebogens}
            \end{figure}


            \item \textbf{(Wunsch) /F100/ Sehen der eigenen Antwortquote\footnote{Antwortquote: der Quotient aus beantworteten Fragen und allen Fragen. (Die Antwortquoten sind also für jeden Proband unterschiedlich.)}}

            \par \textbf{Ziel: }Proband kann im Laufen des Versuchs sehen, ob er so viele Fragen wie erforderlich beantwortet hat
            \par \textbf{Vorbedingung: }Proband hat in der App angemeldet
            \par \textbf{Nachbedingung (Erfolg): }die App zeigt die eigene Antwortquote des Probanden
            \par \textbf{Nachbedingung (Fehlschlag): }keine Antwortquote wird gezeigt
            \par \textbf{Akteure: }Proband selbst
            \par \textbf{Auslösendes Ereignis: }Proband klickt auf “My Answer Rate” in der App
            \par \textbf{Beschreibung: }
                \begin{enumerate}
                    \item Bei Erstellen eines Fragebogens kann der Versuchsleiter eine “Bestehensgrenze” der Antwortquote bestimmen. Ein Proband braucht also mindestens eine bestimmte Menge von Fragen beantworten, um die Vergütung am Ende zu bekommen. Die Standardbestehensgrenze liegt bei 60\%.
                    \item Die Bestehensgrenze und die eigene aktuelle Antwortquote wird nicht in der App gezeigt. Gezeigt wird nur, ob man momentan “besteht” oder “nicht besteht”. Man sieht also keine exakte Zahl.
                    \item Im Laufen des Versuchs wird diesen Status stets aktualisiert. Der Proband kann den eigenen Status immer nachsehen.
                    \item Am Ende eines Versuchs kann der Proband das Smartphone zum Versuchsleiter zeigen. Der Versuchsleiter sieht auf dem Smartphone keinen Proband-ID oder genaue Antwortquote, sondern nur “bestanden” oder “nicht bestanden”. So kann der Versuchsleiter die Vergütung entsprechend zahlen.
                \end{enumerate}

        \end{itemize}

    \chapter{Produktdaten}
        \section{Versuchsleiterdaten}
            \begin{itemize}
                \item \textbf{/D10/} Über einen Versuchsleiter sind folgende Daten zu speichern:
                    \par Name (Anrede, Titel, Vorname, Nachname), Mailadresse (als Kontonummer im System)

                \item \textbf{/D20/} Macht ein Versuchsleiter Versuche, dann sind folgende Daten zu speichern:
                    \par Versuchsummern der zum Versuchsleiter gehörten Versuchen
            \end{itemize}

        \section{Probanddaten}
            \begin{itemize}
                \item \textbf{/D30/} Über einen Proband sind folgende Daten zu speichern:
                    \par Proband-ID, Geburtsdatum, Beruf, Geschlecht, Versuchsnummer des daran beteiligten Versuchs

                \item \textbf{/D40/} Hat ein Proband im Versuch Fragen beantwortet, dann sind folgende Daten zu speichern:
                    \par Antwortquote des Probanden
            \end{itemize}

        \section{Versuchsdaten}
            \begin{itemize}
                \item \textbf{/D50/} Über einen Versuch sind folgende Daten zu speichern:
                    \par Versuchsnummer, Versuchsname, Anfangs- und Endedatum, “Bestehensgrenze” der Antwortquote für die Probanden\footnote{Siehe funktionale Anforderungen F100}, Mailadresse der Versuchsleiter

                \item \textbf{/D60/} Hat ein Versuch Probanden, dann sind folgende Daten zu speichern:
                    \par Proband-IDs

                \item \textbf{/D70/} Hat ein Versuch Fragebogen, dann sind folgende Daten zu speichern:
                    \par Fragebogennummer des Fragebogens
            \end{itemize}

        \section{Fragebogendaten}
            \begin{itemize}
                \item \textbf{/D80/} Über einen Fragebogen sind folgende Daten zu speichern:
                \par Fragebogennummer, Name, Versuchsnummer des zugehörigen Versuchs
                
                \item \textbf{/D90/} Wird ein Fragebogen im Versuch erscheinen, dann sind folgende Daten zu speichern:
                \par Erscheinungsereignisse, Abgabetermin (immer eine Zeit, z.B. 2 Studen nach der Erscheinung des Fragebogens)
                
                \item \textbf{/D100/} Wird einen Fragebogen von Probanden beantwortet, dann sind folgende Daten zu speichern:
                \par Proband-IDs der Antwortgeber, Abgabezeit der Antworten, Anzahl der Antworten, Inhalt der Antworten, Ausschöpfungsquote (response rate)
            \end{itemize}
            
        \section{Fragendaten}
            \begin{itemize}
                \item \textbf{/D110/} Über eine Versuchsfrage sind folgende Daten zu speichern:
                    \par Fragenummer, Fragetyp, Inhalt der Frage, Fragebogennummer des zugehörigen Fragebogens

                \item \textbf{/D120/} Hat eine Frage Folgefragen, dann sind folgende Daten zu speichern:
                    \par Fragenummern der Folgefragen, Bedingung der Folgefragen (Falls man die bestimmte Optionen der Frage wählt, wird die Folgefragen erscheinen.)
            \end{itemize}

   \chapter{Nichtfunktionale Anforderungen}
        \begin{itemize}
            \item /NF10/Reaktionszeit

                \par Die App darf nicht mehr als \_ Sekunden Reaktionszeit haben und die Ladezeit muss unter \_ Sekunden liegen.


        \end{itemize}

        \begin{itemize}
            \item /NF20/Registrierung

                \par Die Registrierung erfolgt unter Angabe einer E-Mail Adresse und einem Passwort.

        \end{itemize}
        \begin{itemize}
            \item /NF30/Passwort

                \par Passwörter müssen mindestens 6-stellig sein. Die App kann angemeldet bleiben.


        \end{itemize}
        \begin{itemize}
            \item /NF40/Globalisierung

                \par Die App ist auf Deutsch und Englisch verfügbar.


        \end{itemize}
        \begin{itemize}
            \item /NF50/Daten speichern

                \par Die Antworten jedes Fragebogens von den Probanden muss ins Handy gespeichert werden.


        \end{itemize}
        \begin{itemize}
            \item /NF60/Größe der App

                \par Die App darf nicht mehr als \_ Mb auf dem Handy benötigen.



        \end{itemize}
        \begin{itemize}
            \item /NF70/Integration ins Google Play Store

                \par Die App soll ins Google Play Store integriert werden.



        \end{itemize}

    \chapter{Benutzungsoberfläche}
        \subsection{Web-Interface}
            \begin{figure}[ht]
                \centering
                \includegraphics[scale = 0.25]{web_set1.png}
                \caption{GUI Entwurf - Questionnaire Settings}
            \end{figure}

	        \begin{figure}[ht]
	        	\centering
	        	\includegraphics[scale = 0.25]{web_set2.png}
	        	\caption{GUI Entwurf - Question Settings}
	        \end{figure}

            \begin{figure}[ht]
                \centering
                \includegraphics[scale = 0.25]{web_answer.png}
                \caption{GUI Entwurf - View Answer}
            \end{figure}

        \newpage
        \subsection{App}
            \vspace*{2cm}

            \begin{figure}[ht]
            	\centering
            	\includegraphics[scale = 0.3]{android_home.jpg}
            	\caption{GUI Entwurf - Homepage}
            \end{figure}

            \vspace*{1cm}
            \begin{figure}[ht]
                \centering
                \includegraphics[scale = 0.3]{android_login.jpg}
                \caption{GUI Entwurf - Login}
            \end{figure}

            \vspace*{1cm}
            \begin{figure}[ht]
                \centering
                \includegraphics[scale = 0.3]{android_answer.jpg}
                \caption{GUI Entwurf - Answer Question}
            \end{figure}	

    \chapter{Globale Testfälle}

        Folgende Funktionssequenzen sind zu \"uberpr\"ufen:

        \section{Web-Interface}

            \begin{itemize}
                \item \textbf{/TF10/ Registrieren des Versuchsleiters}
                    \begin{enumerate}
                        \item \par \textbf{Stand: }Offene Homepage des Web-Interface
                            \par \textbf{Aktion: }Benutzer klickt auf den Button ``Sign up''
                            \par \textbf{Reaktion: }Der browser wird zu der Registrieren-Seite umgeleitet
                        \item \par \textbf{Stand: }Benutzer ist in der Registrieren-Seite
                             \par \textbf{Aktion: }Benutzer gibt eine Email-Adresse und ein Passwort wird ein
                             \par \textbf{Reaktion: }Eine Bestätigungsmail wird nach dieser Email-Adresse geschenkt
                   \end{enumerate}

                    \item \textbf{/TF20/ Anmelden des Versuchsleiters}
                        \begin{enumerate}
                            \item \par \textbf{Stand: }Offene Homepage des Web-Interface
                                \par \textbf{Aktion: }Registrierte Email-Adresse und Passwort eingeben, und auf “Sign in” klicken
                                \par \textbf{Reaktion: }Erfolgreich eingeloggt. Der browser wird zum Dashboard umgeleitet.
                        \end{enumerate}

                \item \textbf{/TF30/ Vergessenes Passwort neu setzen}
                    \begin{enumerate}
                        \item \par \textbf{Stand: }Offene Homepage des Web-Interface
                              \par \textbf{Aktion: }Auf ``Forgot Passwort'' klicken
                              \par \textbf{Reaktion: }Der browser wird zu der Email-Adresse-Eingeben-Seite umgeleitet
                        \item \par \textbf{Stand: }Offene Email-Adresse-Eingeben-Seite
                              \par \textbf{Aktion: }Email-Adresse eingeben, und auf ``Send email'' klicken
                              \par \textbf{Reaktion: }Ein Mail mit einem neuen zufällig generierten Passwort wird zu dieser Email-Adresse geschenkt
                    \end{enumerate}
                    
                \item \textbf{/TF40/ Verwaltung des Fragebogens}
                    \begin{enumerate}
                        \item \par \textbf{Stand: }Kein Fragebogen vorhanden
                              \par \textbf{Aktion: }Der Versuchsleiter klickt auf ``Go to question settings'' Button
                              \par \textbf{Reaktion: }Ein Dialogfeld ``Kein Fragebogen vorhanden.'' erscheint
                        \item \par \textbf{Stand: } Kein Fragebogen vorhanden
                              \par \textbf{Aktion: }Der Versuchsleiter klickt auf ``save'' Button
                              \par \textbf{Reaktion: }Ein Dialogfeld ``Kein Fragebogen vorhanden.'' erscheint
                        \item \par \textbf{Stand: }Kein Fragebogen vorhanden
                              \par \textbf{Aktion: }Der Versuchsleiter klickt auf ``submit'' Button
                              \par \textbf{Reaktion: }Ein Dialogfeld ``Kein Fragebogen vorhanden.'' erscheint
                        \item \par \textbf{Stand: }Kein Fragebogen vorhanden
                              \par \textbf{Aktion: }Der Versuchsleiter klickt auf ``new'' Button
                              \par \textbf{Reaktion: }Der Versuchsleiter kann der Name eines Fragebogens eingeben und dieser Fragebogen wird erstellt
                        \item \par \textbf{Stand: }Ein Fragebogen ausgewählt
                              \par \textbf{Aktion: }Der Versuchsleiter klickt auf ``Go to question settings'' Button
                              \par \textbf{Reaktion: }Der Browser wechselt zur Webseite für die Verwaltung der Fragen
                        \item \par \textbf{Stand: }Ein Fragebogen ausgewählt und eingestellt
                              \par \textbf{Aktion: }Der Versuchsleiter klickt auf ``save'' Button
                              \par \textbf{Reaktion: }Ein Dialogfeld ``Fragebogen wird gespeichert'' erscheint
                        \item \par \textbf{Stand: }Ein Fragebogen gespeichert
                              \par \textbf{Aktion: }Der Versuchsleiter klickt auf ``Refresh'' Button
                              \par \textbf{Reaktion: }Die Voransicht des Fragebogens erscheint
                        \item \par \textbf{Stand: }Alle Einstellung fertig aber keine Frage haben
                              \par \textbf{Aktion: }Der Versuchsleiter klickt auf ``submit'' Button
                              \par \textbf{Reaktion: }Ein Dialogfeld ``Bitte mind. eine Frage erstellen'' erscheint
                        \item \par \textbf{Stand: }Alle Einstellung fertig und Frage vorhanden
                              \par \textbf{Aktion: }Der Versuchsleiter klickt auf ``submit'' Button
                              \par \textbf{Reaktion: }Ein Dialogfeld ``Fragebogen wird hochgeladen'' erscheint
                    \end{enumerate}

                \item \textbf{/TF50/ Verwaltung der Fragen}
                        \begin{enumerate}
                        \item \par \textbf{Stand: } Webseite für die Verwaltung der Fragen ist geladen
                              \par \textbf{Aktion: } Der Versuchsleiter klickt auf ``+'' Button, der links neben den Fragenliste liegt
                              \par \textbf{Reaktion: } Eine neue Frage wird in die Fragenliste hinzugefügt
                              
                        \item \par \textbf{Stand: } Webseite für die Verwaltung der Fragen ist geladen und eine Frage von der Frageliste ausgewählt
                              \par \textbf{Aktion: }  Der Versuchsleiter klickt auf ``-'' Button, der links neben den Fragenliste liegt 
                              \par \textbf{Reaktion: } die ausgewählte Frage wird in die Fragenliste gelöscht

                        \item \par \textbf{Stand: } Eine Frage von der Frageliste ausgewählt
                              \par \textbf{Aktion: } Der Versuchsleiter klickt auf ``Question type'' Combobox
                              \par \textbf{Reaktion: } Er kann eine Fragenart auswählen

                        \item \par \textbf{Stand: }  Eine Frage von der Frageliste ausgewählt
                              \par \textbf{Aktion: } Der Versuchsleiter klickt auf ``+'' Button, der rechts neben das Textfeld ``new choice'' liegt
                              \par \textbf{Reaktion: } Eine neue Option für die Antwort erstellt

                        \item \par \textbf{Stand: } Eine Frage hat Optionen
                              \par \textbf{Aktion: } Der Versuchsleiter klickt auf der Fragenliste, die unten eine Option der Frage liegt
                              \par \textbf{Reaktion: }Der Versuchsleiter kann eine vorhandene Frage auswählen

                        \item \par \textbf{Stand: }  Verwaltung der Fragen fertig
                              \par \textbf{Aktion: } Der Versuchsleiter klickt auf ``Refresh'' Button
                              \par \textbf{Reaktion: } Die Voransicht des Fragebogens erneuet

                        \item \par \textbf{Stand: } Verwaltung der Fragen fertig
                              \par \textbf{Aktion: } Der Versuchsleiter klickt auf ``Go to questionnaire settings'' Button
                              \par \textbf{Reaktion: } Der Browser wechselt zur Webseite für die Verwaltung des Fragebogens        

                    \end{enumerate}


                \item \textbf{/TF60/ Antwortsstatus besichtigen}
                    \begin{enumerate}
                        \item \par \textbf{Stand: } Der Versuchsleiter ist noch nicht angemeldet.
                              \par \textbf{Aktion: } Der Versuchsleiter klickt den Button ``View Answers''
                              \par \textbf{Reaktion: } Es wrid zur Anmeldungsseite wechseln.

                        \item \par \textbf{Stand: } Der Versuchsleiter ist angemeldet.
                              \par \textbf{Aktion: } Der Versuchsleiter klickt den Button ``View Answers''
                              \par \textbf{Reaktion: } Der Versuchsleiter kann die \"ubersicht von allen Antwort sehen.

                        \item \par \textbf{Stand: } Der Versuchsleiter bleibt in der Seite ``View Answers''
                              \par \textbf{Aktion: } Der Versuchsleiter klickt den Button ``Refresh''
                              \par \textbf{Reaktion: } Die neue \"ubersicht von allen Antwort wird gezeigt.

                        \item \par \textbf{Stand: } Der Versuchsleiter bleibt in der Seite ``View Answers''
                              \par \textbf{Aktion: } Der Versuchsleiter gibt einige Bedingungen ein und klickt er den Button ``Confirm''
                              \par \textbf{Reaktion: } Die Antwort wird gefiltert. Die \"ubersicht von Antwort der Probanden, die alle eingegebene Bedingungen erf\"ullen, wird geladen.\"ubersicht
                    \end{enumerate}
                    % \begin{itemize}
                    %     \item Durch Klicken von Button ``View answers'' in der linke Men\"u siehe der Versuchsleiter die \"Ubersicht von allen Antwort des aktuellen Fragebogens. \\
                    %     (Abbildung \ref*{web_ViewAnswer})
                    %     \item Durch Klicken von Button ``Refresh'' wird die neuese \"Ubersicht geladen.
                    %     \item Nach Eingeben von verschiedenene Kriterien wird die \"Ubersicht von Antwort der bestimmten Probanden gezeigt.
                    % \end{itemize}

                \item \textbf{/TF65/ Feedback (Motivation) senden}

                    \begin{enumerate}
                        \item \par \textbf{Stand: } Der Versuchsleiter ist noch nicht angemeldet.
                              \par \textbf{Aktion: } Der Versuchsleiter klickt den Button ``Send Motivation / Feedback''
                              \par \textbf{Reaktion: } Es wrid zur Anmeldungsseite wechseln.
                        \item \par \textbf{Stand: } Der Versuchsleiter ist angemeldet.
                              \par \textbf{Aktion: }  Der Versuchsleiter klickt den Button ``Send Motivation / Feedback''
                              \par \textbf{Reaktion: } Der Versuchsleiter wird zur Seite hergeleitet, wo er Motivation / Feedback schreiben und senden kann.
                        \item \par \textbf{Stand: } Der Versuchsleiter bleibt in der Seite ``Send Motivation / Feedback''
                              \par \textbf{Aktion: } Der Versuchsleiter hat die Motivation fertig geschrieben und klickt er den Button ``send''.
                              \par \textbf{Reaktion: } Die Smartphone von Proband erh\"alt eine Notifikation. Durch Klicken von dieser Notifikation kann der Proband die Motivation sehen.
                        % \item \par \textbf{Stand: }
                        %       \par \textbf{Aktion: }
                        %       \par \textbf{Reaktion: }
                        % \item \par \textbf{Stand: }
                        %       \par \textbf{Aktion: }
                        %       \par \textbf{Reaktion: }
                    \end{enumerate}


                    % \par Durch Klicken von Button ``Send Motivation / Feedback'' sieht der Versuchsleiter die Seite, wo er Motivation und Feedback schreiben und senden kann.
                    % \par Durch Klicken von Button ``send'' wird die geschriebene Motivation zu allen Probanden gesandt.
                    % \par Die Smartphone von Proband erh\"alt eine Notifikation. Durch Klicken von dieser Notifikation kann der Proband die Motivation sehen.

                \item \textbf{/TF70/ Exportieren von Daten}
                \begin{itemize}
                    \item \par \textbf{Stand: }Sein angemeldet
                          \par \textbf{Aktion: }Klicken auf ``Export data''
                          \par \textbf{Reaktion: }Daten sind exportiert
                \end{itemize}


            \end{itemize}


        \vspace*{2cm}
        \section{Android-Anwendung}

            \begin{itemize}

            \item \textbf{/TF-/ Erstes Anmelden der Probanden}
            \begin{enumerate}
                \item \par \textbf{Stand: }Der Proband hat sich noch nicht in der App angemeldet
                \par \textbf{Aktion: }Der Proband öffnet die App
                \par \textbf{Reaktion: }Die App wechselt auf die ``User Login'' Seite
                \item \par \textbf{Stand: }Die ``User Login'' Seite liegt vor
                \par \textbf{Aktion: }Der Proband gibt Versuch-ID und die angeforderte Information ein, und klickt den Button ``Submit''
                \par \textbf{Reaktion: }Der Proband wird auf die Hauptseite weitergeleitet
            \end{enumerate}

	        \item \textbf{/TF-/ Automatisches Anmelden der Probanden}
	        \begin{enumerate}
	        	\item \par \textbf{Stand: }Der Proband hat sich schon einmal in der App angemeldet
	        	\par \textbf{Aktion: }Der Proband öffnet die App
	        	\par \textbf{Reaktion: }Der Proband wird automatisch auf die Hauptseite weitergeleitet
	        \end{enumerate}

	        \item \textbf{/TF-/ Antworten eines Fragebogens}
	        \begin{enumerate}
	        	\item \par \textbf{Stand: }Die Hauptseite der App liegt vor
	        	\par \textbf{Aktion: }Der Proband wählt einen Fragebogen aus ``My Questionnaires''
	        	\par \textbf{Reaktion: }Der Proband wird auf die erste Frage des gewählten Fragebogens weitergeleitet
	        	\item \par \textbf{Stand: }Der Proband ist auf der Seite einer Frage
	        	\par \textbf{Aktion: }Der Proband beantwortet die Frage und klickt den Button ``Next Question''
	        	\par \textbf{Reaktion: }Der Proband wird auf die nächste Frage des Fragebogens weitergeleitet
	        	\item \par \textbf{Stand: }Der Proband ist auf der Seite einer Frage
	        	\par \textbf{Aktion: }Der Proband klickt den Button ``Homepage''
	        	\par \textbf{Reaktion: }Der Proband wird auf die Hauptseite weitergeleitet und die schon eingegebenen Antworten des Fragebogens werden nicht gespeichert
	        	\item \par \textbf{Stand: }Der Proband ist auf der Seite letzter Frage des Fragebogens
	        	\par \textbf{Aktion: }Der Proband beantwortet die Frage und klickt den Button ``Submit''
	        	\par \textbf{Reaktion: }Die App zeigt ``You have successfully submitted your answers!'' und wechselt auf die Hauptseite
	        \end{enumerate}

            \item \textbf{/TF--/ Zeigen einer bestandenen Antwortquote}
            \begin{enumerate}
                \item \par \textbf{Stand: }Die Bestehensgrenze der Antwortquote liegt bei 60\%. Der Proband bekommt 10 Fragen.
                \par \textbf{Aktion: }Der Proband beantwortet 6 Fragen, dann klickt auf “My Response Rate”.
                \par \textbf{Reaktion: }Die App zeigt “You are currently above the pass mark. Very good! Please go on!”
            \end{enumerate}

            \item \textbf{/TF--/ Zeigen einer nicht bestandenen Antwortquote}
            \begin{enumerate}
                \item \par \textbf{Stand: }Die Bestehensgrenze der Antwortquote liegt bei 60\%. Der Proband bekommt 10 Fragen.
                \par \textbf{Aktion: }Der Proband beantwortet 5 Fragen, dann klickt auf “My Response Rate”.
                \par \textbf{Reaktion: }Die App zeigt ““You are currently below the pass mark. No problem, you can just improve that!”
            \end{enumerate}

            \end{itemize}

    \chapter{Produktübersicht}
        % \section{Anwendungsfalldiagramm}
            \begin{figure}[htbp]
                \centering
                \includegraphics[scale = 0.4]{UseCaseDiagram1.jpg}
                \caption{Anwendungsfalldiagramm}
            \end{figure}

%    \chapter{Qualitätsziele}
%        Qualiätsziele: Allgemeine Ziele sind meistens Änderbarkeit und Wartbarkeit.
%        Ziele sollten jedoch grundsätzlich messbar, spezifisch und relevant sein.

    \glsaddall
    \printglossary

    % Abbildungsverzeichnis
    \listoffigures

\end{document}
