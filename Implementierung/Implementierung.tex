\documentclass[a4paper]{scrreprt}

\usepackage[german]{babel}
\usepackage[utf8]{inputenc}
\usepackage[T1]{fontenc}
\usepackage{ae}
\usepackage[bookmarks,bookmarksnumbered]{hyperref}
\usepackage{graphicx}
\usepackage[toc]{glossaries}
\usepackage{float}
\graphicspath{ {Images/} }
\setcounter{secnumdepth}{5}
\makeglossaries

\begin{document}
    \newglossaryentry{Studienleiter}{name=Studienleiter, description={Person, unter dessen Leitung eine Studie durchgeführt wird}}
    \newglossaryentry{Proband}{name=Proband, description={Leute, der sich an einer Studie beteiligt}}
    \newglossaryentry{Web-Interface}{name=Web-Interface, description={Benutzeroberfläche, die man in einem Web-Browser nutzen kann}}
    \newglossaryentry{Antwortsstatus}{name=Antwortsstatus, description={Die statistischen Daten der Antworten von Probanden innerhalb von G\"ultigkeitszeitbereich des Fragebogens}}
    \newglossaryentry{Android-App}{name=Android-App, description={Anwendungssoftware für Mobilgeräte mit Android als Betriebssystem}}
    \newglossaryentry{Erscheinungsereignis}{name={Erscheinungsereignis},description={ein Ereignis, das allein oder mit anderen Ereignissen zussamen einen Fragebogen auslösen kann}}
    %\newglossaryentry{Aussch\"opfungsquote}{name=Aussch\"opfungsquote, description={der Quotient aus die Anzahl der Probanden, die eine Frage tatsächlich beantworteten und die Anzahl aller Probanden (Die Ausschöpfungsquoten sind also global und für alle Probanden gleich.)}}

    \begin{flushright}
        \includegraphics[scale = 0.7]{kit-logo.jpg}\\[0.5cm]
        % \includegraphics[scale = 1]{teco.jpg}
    \end{flushright}
    % \includegraphics[scale = 0.5]{kit-logo.jpg} \hspace{4cm} \includegraphics[scale = 1]{teco.jpg}
    \vspace*{2cm}

    \begin{center} \large

        Praxis der Softwareentwicklung
        \vspace * {1.5cm}

        \textbf{\huge Mind Rate}

        \vspace*{1cm}


        {\Large Ein interaktives System mit Android-Client f\"ur Studien nach Experience-Sampling-Method (ESM)}

        \vspace*{1cm}

        \textbf{\Large Implementierung}
        \vspace*{2cm}

        Shanshan Du, Yi Ge, Renhan Lou, Ruoheng Ma, Haobin Tan
        \vspace*{1cm}

        19. Februar 2017
        \vspace*{2.5cm}

        Betreuung: Anja Exler, Dr. Andrea Schankin\\[0.5cm]
        Forschungsgruppe TECO: Technology for Pervasive Computing\\[0.5cm]

        Karlsruher Institut für Technologie
    \end{center}
    \thispagestyle{empty}

    \tableofcontents

    \chapter{Einleitung}

    
    


    
    \newpage
    \chapter{Änderungen am Entwurf}
        

        \section{Android Applikation}
            

        \section{Server}
            
       
        \newpage
    \chapter{Implementierte Muss- und Wunschkriterien}
       

        \section{Android Applikation}
            
            
            
      

        \section{Server}
           
           
           \newpage
    \chapter{GANTT Diagramm}
        
        % \section{\gls{Web-Interface}}
        \section{Android Applikation}
            
        \section{Server}

       \newpage   

    \chapter{Übersicht zu unit tests}
        
        \section{Android Applikation}
            
        \section{Server}
         
    \newpage 

\end{document}
